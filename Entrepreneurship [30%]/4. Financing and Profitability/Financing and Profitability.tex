% section | subsection | subsubsection | paragraph | subparagraph | verbatim | enumerate | item

% Just let it be an Article bro
\documentclass{article}

% Cover Metadata
\title{Financing and Profitability}
\date{12 Oktober 2025}
\author{Maulana Hafidz Ismail}

% Equation Setup
\usepackage{amsmath}

% Image Setup
\usepackage{graphicx}
\usepackage{float}

% Link Setup
\usepackage{hyperref}

% Highlighting Setup
\usepackage{soul}
\usepackage{xcolor}
\sethlcolor{red!30}
\newcommand{\hlblue}[1]{\sethlcolor{cyan!30}\hl{#1}\sethlcolor{yellow}}

% Line Width
\sloppy
\setlength{\emergencystretch}{3em}
\hbadness=99999

% Code Syntax Setup
\usepackage{listings}
\usepackage{xcolor} % untuk warna
\lstset{
basicstyle=\ttfamily\small,
frame=single,
breaklines=true,
backgroundcolor=\color{gray!10},
keywordstyle=\color{blue},
commentstyle=\color{gray},
stringstyle=\color{red},
tabsize=2,
captionpos=b
}

% Start of the Article
\begin{document}

% Cover Page
\pagenumbering{gobble}
\maketitle
\newpage

% Table of Content Page
\tableofcontents
\newpage
\pagenumbering{arabic}

% Main Content
\section{Business Models and Keeping Customers}

\subsection{KEY QUESTIONS}
\begin{itemize}
    \item What is a sustainable business model?
    \item How much capital do I need?
    \item How can I estimate when I'm going to run out of cash  or when I will reach positive cash flow?
    \item What are the available sources of capital?
    \item What financial analyses do I need for fund raising?
    \item What other information do investors typically require?
    \item When I succeed, how can I extract the financial value from my business?
\end{itemize}

\subsection{Business Models sebagai Narasi}
\begin{itemize}
    \item Business model berfungsi sebagai cerita yang menjelaskan bagaimana bisnis Anda dapat berhasil. Ini menggabungkan angka dan narasi untuk menunjukkan potensi kesuksesan.
    \item Contoh: WebVan memiliki narasi yang masuk akal untuk pengiriman bahan makanan, tetapi gagal karena biaya yang tinggi. Sebaliknya, Priceline memiliki narasi yang kurang masuk akal tetapi juga mengalami kerugian besar.
\end{itemize}

\subsection{Business Models sebagai Checklist}
\begin{itemize}
    \item Business models juga berfungsi sebagai checklist untuk membantu Anda memikirkan bagaimana bisnis Anda akan beroperasi dan menghasilkan uang.
    \item Pertanyaan penting yang harus dipertimbangkan:
          \begin{enumerate}
              \item What’s your value proposition? (Apa nilai yang Anda tawarkan kepada pelanggan?)
              \item How will you build your product or service? (Bagaimana Anda akan membangun produk atau layanan Anda?)
              \item How will you bring it to market? (Bagaimana Anda akan memasarkan produk atau layanan Anda?)
              \item How will you make money? (Bagaimana Anda akan menghasilkan uang?)
          \end{enumerate}
\end{itemize}

\subsection{Tipe-tipe Business Models}
\begin{itemize}
    \item Architecture Business Model: Model bisnis yang menghubungkan berbagai pihak, seperti eBay dan Uber, yang menghasilkan uang dari setiap transaksi tanpa memiliki produk.
    \item Disruptive Model: Model yang mencoba mengalahkan pesaing dengan menawarkan produk atau layanan yang lebih baik dan lebih murah, seperti Warby Parker yang menjual kacamata dengan harga lebih rendah.
    \item Value Chain Model: Model yang berfokus pada satu bagian dari rantai nilai dan melakukannya lebih baik daripada pesaing, seperti Peapod yang mengelola pengiriman bahan makanan.
\end{itemize}

\subsection{Pengertian Customer Lifetime Value (CLV)}
\begin{itemize}
    \item \textbf{Customer Lifetime Value (CLV)}: Nilai total yang diharapkan dari seorang pelanggan selama masa hubungan mereka dengan perusahaan.
    \item Definisi Istilah Teknis
          \begin{itemize}
              \item Annual Contribution: Jumlah uang yang dihasilkan dari pelanggan setiap tahun.
              \item Churn Rate: Persentase pelanggan yang berhenti menggunakan layanan dalam periode tertentu.
              \item Growth Rate: Persentase peningkatan kontribusi dari pelanggan setiap tahun.
              \item Discount Rate: Tingkat pengembalian yang diharapkan dari investasi, yang digunakan untuk menghitung nilai waktu dari uang.
          \end{itemize}
\end{itemize}

\subsection{Mengukur CLV}
\begin{itemize}
    \item CLV dihitung dengan mempertimbangkan beberapa faktor, yaitu kontribusi tahunan, churn rate, discount rate, dan growth rate. Rumus umumnya adalah:
          \begin{align*}
              CLV = \frac{\text{Annual Contribution}}{\text{Churn Rate} + \text{Discount Rate} - \text{Growth Rate}}
          \end{align*}
    \item Ada juga rumus cepatnya yang disebut Quick CLV atau QCLV, namun tidak seakurat CLV:
          \begin{align}
              QCLV = \frac{\text{Annual Contribution}}{\text{Churn Rate}}
          \end{align}
    \item Contoh Kasus: Sebuah perusahaan jasa keuangan yang kehilangan 20\% pelanggan setiap tahun, tetapi pelanggan yang bertahan meningkatkan kontribusi mereka sebesar 5\% setiap tahun.
    \item Langkah Pengukuran: Menggunakan data terbatas untuk memperkirakan nilai pelanggan dengan asumsi bahwa pelanggan memberikan kontribusi tahunan yang tetap.
\end{itemize}

\subsection{Faktor yang Mempengaruhi CLV}
\begin{itemize}
    \item \textbf{Annual Contribution}: Kontribusi tahunan dari pelanggan, misalnya \$250.
    \item \textbf{Churn Rate}: Tingkat kehilangan pelanggan, misalnya 20\%.
    \item \textbf{Growth Rate}: Tingkat pertumbuhan kontribusi tahunan, misalnya 5\%.
    \item \textbf{Discount Rate}: Tingkat diskonto atau nilai waktu dari uang, misalnya 10\%.
\end{itemize}

\subsection{Contoh Perhitungan CLV}
\begin{itemize}
    \item Kasus Pertama: Jika pelanggan memberikan \$250 setiap tahun tanpa kehilangan, CLV dihitung sebagai \$2,500.
    \item Kasus Kedua: Dengan 20\% kemungkinan kehilangan pelanggan, CLV menjadi sekitar \$833.
    \item Kasus Ketiga: Jika kontribusi meningkat 5\% setiap tahun, CLV menjadi sekitar \$1,000.
\end{itemize}

\subsection{Aplikasi CLV dalam Keputusan Bisnis}
\begin{itemize}
    \item Evaluasi Upaya Pemasaran: Perbandingan biaya akuisisi pelanggan dengan nilai pelanggan yang diharapkan.
    \item Strategi Retensi: Menghitung apakah mengurangi churn rate dari 20\% menjadi 18\% akan meningkatkan CLV secara signifikan.
    \item Perbandingan Pelanggan: Menggunakan CLV untuk membandingkan nilai pelanggan yang berbeda, seperti pelanggan prepaid dan postpaid dalam industri telekomunikasi.
\end{itemize}


\newpage

\section{Financing, Valuation, and Terms}
\subsection{Metode Pembiayaan}
\begin{itemize}
    \item \textbf{Dilutive Funding}: Pembiayaan yang mengharuskan pemilik bisnis memberikan sebagian ekuitas perusahaan kepada investor. Contoh: angel investors dan venture capital.
    \item \textbf{Non-Dilutive Funding}: Pembiayaan yang tidak memerlukan pengorbanan ekuitas, seperti self-funding dan debt funding.
\end{itemize}

\subsection{Sumber Pembiayaan Umum}
\begin{itemize}
    \item \textbf{Self-Funding}: Menggunakan tabungan pribadi untuk membiayai bisnis. Penting untuk menetapkan anggaran realistis.
    \item \textbf{Debt Funding}: Meminjam uang dari lembaga keuangan tanpa memberikan ekuitas. Contoh: pinjaman dari pemerintah atau bank.
    \item \textbf{Friends and Family Funding}: Meminta dukungan finansial dari orang terdekat. Ini bisa rumit karena ada komponen emosional.
\end{itemize}

\subsection{Tahapan Pembiayaan}
\begin{itemize}
    \item \textbf{Pre-Seed Stage}: Pembiayaan awal untuk mengembangkan produk minimal (minimal viable product).
    \item \textbf{Seed Funding}: Pembiayaan untuk menunjukkan minat pasar dan membangun prototipe.
    \item \textbf{A Round Funding}: Pembiayaan untuk mulai mengembangkan bisnis secara serius, biasanya antara $3 hingga $5 juta.
\end{itemize}

\subsection{Investor dan Pendekatan Pembiayaan}
\begin{itemize}
    \item \textbf{Angel Investors}: Individu yang berinvestasi di tahap awal, biasanya antara \$25,000 hingga jutaan dolar.
    \item \textbf{Venture Capital (VC)}: Investasi besar yang biasanya terjadi di tahap A atau lebih lanjut, dengan dana yang sering kali mencapai miliaran dolar.
    \item \textbf{Crowdfunding}: Mengumpulkan dana dari banyak orang, bisa bersifat dilutive atau non-dilutive.
\end{itemize}

\subsection{Sumber Pembiayaan untuk Startup}
\begin{itemize}
    \item \textbf{Founder Equity}: Rata-rata investasi dari pendiri adalah sekitar \$46,000, yang bisa berupa uang tunai atau nilai dari kerja mereka.
    \item \textbf{Debt}: Sumber pembiayaan kedua terbesar adalah utang, dengan rata-rata sekitar \$41,000.
    \item \textbf{Venture Capital (VC)}: Pembiayaan dari VC adalah sekitar \$31,000, yang merupakan sumber ketiga terbesar.
    \item \textbf{Friends and Family}: Pembiayaan dari teman dan keluarga rata-rata sekitar \$10,600.
\end{itemize}

\subsection{Rundown Pembiayaan}
\begin{itemize}
    \item \textbf{Rounds}: Pembiayaan startup biasanya terjadi dalam "rounds" yang merupakan pengalaman penggalangan dana yang terpisah. Setiap round dapat diberi label A, B, C, dan seterusnya.
    \item \textbf{Convertible Debt}: Ini adalah utang yang dapat dikonversi menjadi ekuitas saat perusahaan melakukan penggalangan dana resmi. Ini memungkinkan penundaan penentuan nilai perusahaan hingga investor profesional terlibat.
\end{itemize}

\subsection{Sumber Pembiayaan Utama}
\begin{itemize}
    \item \textbf{Venture Capital}: Melibatkan mitra terbatas yang menginvestasikan uang ke dalam dana yang kemudian diinvestasikan ke startup. VC biasanya mencari investasi di atas \$2 juta dan memberikan bimbingan serta koneksi.
    \item \textbf{Angel Investors}: Individu kaya yang berinvestasi di perusahaan, biasanya di tahap awal dengan investasi antara \$25,000 hingga \$500,000.
    \item \textbf{Super Angels}: Angel investors yang sangat terhubung dan sering membuat banyak investasi kecil, memberikan sinyal kualitas untuk perusahaan.
    \item \textbf{Seed Accelerators}: Program yang mendukung startup awal dengan bimbingan dan dana kecil, biasanya mengambil 5-10\% ekuitas.
    \item \textbf{Crowdfunding}: Metode penggalangan dana dari banyak orang, bisa dari \$100 hingga lebih dari \$1,000,000.
\end{itemize}

\subsection{Pentingnya Valuasi}
\begin{itemize}
    \item Valuasi adalah proses menentukan nilai perusahaan sebelum menerima investasi. Ini penting untuk menjawab pertanyaan: "Jika saya menginvestasikan uang saya, apa yang saya dapatkan sebagai imbalan?"
    \item Nilai ditentukan melalui negosiasi antara investor dan pengusaha, dan mencerminkan aliran kas yang diharapkan di masa depan.
\end{itemize}

\subsection{Pre-Money dan Post-Money Valuation}
\begin{itemize}
    \item \textbf{Pre-Money Valuation}: Nilai perusahaan sebelum investasi. Contoh: Jika nilai perusahaan ditetapkan sebesar \$1 juta.
    \item \textbf{Post-Money Valuation}: Nilai perusahaan setelah investasi. Contoh: Jika investor memberikan \$100 ribu, maka nilai perusahaan menjadi \$1,1 juta.
\end{itemize}

\subsection{Metode Penilaian Valuasi}
\begin{enumerate}
    \item Cost Approach: Menghitung biaya untuk mereplikasi apa yang telah dibuat. Ini berguna untuk perusahaan tahap awal.
          \begin{itemize}
              \item Definisi: Menghitung nilai berdasarkan biaya yang dikeluarkan untuk menciptakan produk atau layanan.
          \end{itemize}
    \item Comparable Transactions: Membandingkan dengan transaksi serupa di industri yang sama. Ini membantu menentukan nilai yang wajar.
          \begin{itemize}
              \item Definisi: Menggunakan data dari transaksi lain untuk menentukan nilai perusahaan.
          \end{itemize}
    \item Probability Weighted Expected Return Method (PWERM): Menghitung nilai berdasarkan skenario masa depan dan probabilitasnya. Ini melibatkan penilaian risiko.
          \begin{itemize}
              \item Definisi: Metode yang menghitung nilai berdasarkan berbagai kemungkinan hasil di masa depan dan peluang terjadinya.
          \end{itemize}
    \item Earnings Multiple: Menggunakan pendapatan perusahaan untuk menentukan nilai. Ini lebih relevan untuk perusahaan yang sudah beroperasi.
          \begin{itemize}
              \item Definisi: Menghitung nilai perusahaan berdasarkan penghasilan yang dihasilkan, biasanya menggunakan rasio tertentu.
          \end{itemize}
\end{enumerate}

\subsection{Menunda Penilaian}
\begin{itemize}
    \item Convertible Note: Sebuah instrumen yang memungkinkan investor untuk memberikan pinjaman yang dapat dikonversi menjadi ekuitas di masa depan, biasanya pada penilaian yang lebih baik.
    \item Definisi: Surat utang yang dapat diubah menjadi saham perusahaan di masa depan.
\end{itemize}

\subsection{Equity Financing dan Term Sheets}
\begin{itemize}
    \item \textbf{Equity Financing} adalah cara untuk mendapatkan dana dengan menjual saham perusahaan kepada investor. Ini melibatkan banyak istilah dan dokumen hukum yang perlu dipahami.
    \item \textbf{Term Sheet} adalah dokumen yang berisi tawaran dari investor, seperti venture capitalist, yang menjelaskan syarat-syarat investasi.
\end{itemize}

\subsection{Diluation dan Valuasi}
\begin{itemize}
    \item Dilution terjadi ketika pemilik saham yang ada memiliki persentase kepemilikan yang lebih kecil setelah investasi baru. Misalnya, jika dua pendiri memiliki 100\% saham dan investor baru masuk, persentase kepemilikan pendiri akan berkurang.
    \item Valuation adalah penilaian nilai perusahaan. Ada dua jenis valuasi:
          \begin{itemize}
              \item Pre-Money Valuation: Nilai perusahaan sebelum investasi baru.
              \item Post-Money Valuation: Nilai perusahaan setelah investasi baru ditambahkan.
          \end{itemize}
\end{itemize}

\subsection{Contoh Proses Investasi}
\begin{itemize}
    \item Dalam contoh, seorang Angel Investor menginvestasikan \$2 juta ke perusahaan yang memiliki Pre-Money Valuation sebesar \$8 juta. Setelah investasi, Post-Money Valuation menjadi \$10 juta.
    \item \textbf{Option Pool} adalah bagian dari saham yang disisihkan untuk memberikan opsi saham kepada karyawan. Dalam contoh ini, investor meminta 15\% dari perusahaan untuk Option Pool.
\end{itemize}

\subsection{Dampak pada Kepemilikan}
\begin{itemize}
    \item Setelah investasi dan penciptaan Option Pool, kepemilikan pendiri akan berkurang. Misalnya, pendiri satu yang awalnya memiliki 40\% menjadi 26\%, dan pendiri dua dari 60\% menjadi 39\%.
    \item Meskipun persentase kepemilikan berkurang, nilai total saham pendiri bisa meningkat karena valuasi perusahaan yang lebih tinggi.
\end{itemize}

\subsection{Likuidasi dan Preferensi}
\begin{itemize}
    \item \textbf{Liquidation Preference} adalah hak investor untuk mendapatkan kembali investasi mereka sebelum pemegang saham lainnya jika perusahaan dijual. Ada dua jenis:
          \begin{itemize}
              \item Participating: Investor mendapatkan kembali investasi mereka dan juga berbagi keuntungan.
              \item Non-Participating: Investor hanya mendapatkan kembali investasi mereka atau berbagi keuntungan, tergantung mana yang lebih menguntungkan.
          \end{itemize}
\end{itemize}

\newpage

\section{Private and Public Financing, and Calculating Breakeven}

\subsection{Pendanaan untuk Startup}
\begin{itemize}
    \item Venture Capital adalah salah satu bentuk pembiayaan yang banyak digunakan oleh startup yang sukses, seperti Google dan Facebook.
    \item Selain venture capital, ada alternatif lain seperti:
          \begin{itemize}
              \item Commercial Bank Loan: Pinjaman dari bank.
              \item Angel Investors: Individu kaya yang memberikan dana untuk startup.
              \item Crowdfunding: Pendanaan dari banyak orang melalui platform online.
              \item Government Grants: Hibah dari pemerintah untuk penelitian dan pengembangan.
              \item Bootstrapping: Pendanaan dari pendapatan yang dihasilkan sendiri.
          \end{itemize}

\end{itemize}

\subsection{Proses Pendanaan dan Contoh}
\begin{itemize}
    \item Contoh perusahaan yang menerima venture capital adalah Apple dan Genentech.
    \item Proses pendanaan menunjukkan bahwa seiring berkembangnya perusahaan, harga per saham yang dibayar oleh investor meningkat.
\end{itemize}

\subsection{Masalah Asimetri Informasi}
\begin{itemize}
    \item Asymmetric Information: Ketidakseimbangan informasi antara entrepreneur dan investor. Entrepreneur biasanya memiliki informasi lebih baik tentang peluang sukses dibandingkan investor.
    \item Hidden Information: Masalah di mana entrepreneur tidak mengungkapkan semua masalah yang ada pada perusahaan kepada investor.
    \item Hidden Action: Setelah kontrak ditandatangani, investor tidak dapat memantau penggunaan dana oleh entrepreneur.
\end{itemize}

\subsection{Aktivitas Venture Capitalists}
\begin{itemize}
    \item Venture Capitalists melakukan beberapa aktivitas untuk mendapatkan kompensasi, termasuk:
          \begin{itemize}
              \item Mengumpulkan dana dari investor institusi.
              \item Mencari peluang investasi dan melakukan due diligence. Due Diligence adalah proses penyelidikan untuk menilai potensi investasi.
              \item Menyusun dan memantau investasi.
              \item Mengelola portofolio dan membantu perusahaan dalam pengembangan.
          \end{itemize}
\end{itemize}

\subsection{Statistik dan Kinerja}
\begin{itemize}
    \item Dari 22,000 investasi yang dilakukan oleh venture capitalists, 75\% tidak menghasilkan keuntungan, menunjukkan bahwa bisnis venture capital tidak mudah.
    \item Internal Rate of Return (IRR): Ukuran untuk menilai kinerja investasi.
\end{itemize}

\subsection{Penelitian dan Hubungan dengan Kesuksesan Startup}
\begin{itemize}
    \item Penelitian menunjukkan bahwa reputasi venture capitalists berhubungan dengan kesuksesan startup. Entrepreneur cenderung menerima tawaran dari venture capitalists yang memiliki reputasi baik meskipun tawaran tersebut tidak selalu yang terbaik secara finansial.
\end{itemize}

\subsection{Kesimpulan terkait Venture Capitalist}
\begin{itemize}
    \item Venture Capital adalah industri yang kompleks dengan banyak tantangan, terutama terkait dengan asimetri informasi. Namun, industri ini berusaha untuk memberikan nilai tambah bagi startup melalui pendanaan dan layanan tambahan.
\end{itemize}

\subsection{Pengantar Crowdfunding}
\begin{itemize}
    \item Crowdfunding adalah cara untuk mengumpulkan dana dari banyak orang melalui internet, memungkinkan pendiri startup untuk mendapatkan dana tanpa harus memberikan ekuitas kepada investor besar.
    \item Ada empat jenis crowdfunding: \textbf{equity crowdfunding, reward-based crowdfunding, peer-to-peer lending, dan charity crowdfunding}.
\end{itemize}

\subsection{Jenis-jenis Crowdfunding yang dapat dimanfaatkan Entrepreneur}
\begin{itemize}
    \item Equity Crowdfunding: Mengumpulkan dana dengan memberikan persentase kepemilikan perusahaan kepada investor. Ini adalah metode yang relatif baru dan memiliki regulasi yang kompleks.
    \item Reward-based Crowdfunding: Mengumpulkan dana dengan menawarkan produk atau hadiah kepada pendukung. Contoh platformnya adalah Kickstarter dan Indiegogo.
\end{itemize}

\subsection{Manfaat Crowdfunding}
\begin{itemize}
    \item Crowdfunding tidak hanya tentang uang; manfaat lainnya termasuk:
          \begin{itemize}
              \item Mengidentifikasi basis pelanggan.
              \item Membangun komunitas yang tertarik pada produk.
              \item Mendapatkan perhatian media.
          \end{itemize}
\end{itemize}

\subsection{Komunitas dan Crowdfunding}
\begin{itemize}
    \item Keberhasilan crowdfunding sangat bergantung pada kemampuan untuk mengaktifkan komunitas online yang peduli dengan produk yang ditawarkan.
    \item Contoh: Film dokumenter tentang BronyCon berhasil mengumpulkan lebih dari \$300,000 karena komunitas yang mendukungnya.
\end{itemize}

\subsection{Kualitas dan Keberhasilan}
\begin{itemize}
    \item Penelitian menunjukkan bahwa pendukung crowdfunding mencari kualitas dalam proyek yang mereka danai, mirip dengan investor tradisional.
    \item Proyek yang disetujui oleh baik komunitas maupun ahli cenderung lebih sukses.
\end{itemize}

\subsection{Tips untuk Crowdfunding}
\begin{itemize}
    \item Tetapkan tujuan pendanaan yang realistis; semakin besar tujuan, semakin kecil kemungkinan untuk berhasil.
    \item Kenali audiens target dan bangun jaringan yang kuat untuk meningkatkan peluang keberhasilan.
    \item Siapkan kampanye dengan baik, termasuk video berkualitas tinggi dan pembaruan rutin kepada pendukung.
\end{itemize}

\subsection{Kesimpulan terkait Crowdfunding}
\begin{itemize}
    \item \textbf{Crowdfunding} adalah metode yang menarik untuk mengumpulkan dana, terutama untuk produk yang ditujukan kepada konsumen. Namun, penting untuk memahami komunitas dan kualitas proyek yang ditawarkan.
\end{itemize}

\subsection{Debt Financing}
\begin{itemize}
    \item Definisi: Debt financing adalah ketika perusahaan meminjam uang dan memiliki kewajiban untuk membayar kembali pokok dan bunga, tanpa memberikan kepemilikan pada investor.
    \item Risiko: Debt dianggap lebih aman bagi investor dibandingkan equity karena dibayar terlebih dahulu sebelum pemegang saham equity.
\end{itemize}

\subsection{Jenis-jenis Debt Financing}
\begin{itemize}
    \item Trade Debt: Utang yang timbul ketika pemasok memberikan barang dan perusahaan memiliki waktu untuk membayar, biasanya 30-60 hari.
    \item Credit Card Balance: Utang yang dihasilkan dari penggunaan kartu kredit, di mana bunga dikenakan jika saldo tidak dibayar tepat waktu.
\end{itemize}

\subsection{Collateral}
\begin{itemize}
    \item Collateral adalah aset yang digunakan untuk menjamin utang. Jika perusahaan gagal membayar, kreditur dapat mengambil alih aset tersebut.
    \item Contoh: Dalam pinjaman untuk membeli mesin, mesin itu sendiri menjadi collateral.
\end{itemize}

\subsection{Working Capital}
\begin{itemize}
    \item Working capital adalah uang yang diperlukan untuk menjalankan operasi bisnis sehari-hari, meskipun bisnis tersebut menguntungkan.
    \item Contoh: Persediaan barang dan piutang dari pelanggan yang belum dibayar.
\end{itemize}

\subsection{Government Loans}
\begin{itemize}
    \item Definisi: Pinjaman yang diberikan oleh pemerintah untuk mendukung usaha kecil, sering kali dengan syarat yang lebih menguntungkan.
    \item Contoh: SBA (Small Business Administration) di AS yang menjamin pinjaman untuk usaha kecil.
\end{itemize}

\subsection{Burn Rate}
\begin{itemize}
    \item Burn Rate adalah jumlah net negative cash flow per unit waktu, Ini berarti bahwa Burn Rate adalah jumlah uang yang hilang setiap bulan setelah memperhitungkan semua pendapatan dan pengeluaran. Biasanya dinyatakan dalam bulanan. Contohnya, jika burn rate Anda adalah \$30,000 per bulan, itu berarti Anda kehilangan \$30,000 setiap bulan.
    \item Burn rate menunjukkan seberapa cepat perusahaan menghabiskan uang yang dimiliki, dan penting untuk mengetahui berapa lama perusahaan dapat bertahan dengan dana yang ada.
    \item Contoh: Jika perusahaan memiliki pengeluaran bulanan sebesar \$50,000 dan pendapatan sebesar \$20,000, maka Burn Rate-nya adalah:
          \begin{align*}
              \text{Burn Rate} = \text{Pengeluaran} - \text{Pendapatan} \\
              \$50,000 - \$20,000 = \$30,000 \text{ per bulan}.
          \end{align*}
    \item Jika pengeluaran tidak lebih besar dari pendapatan, maka Burn Rate tidak akan menjadi 0, tetapi bisa menjadi negatif atau bahkan positif.
\end{itemize}

\subsection{Runway}
\begin{itemize}
    \item Runway adalah jumlah uang yang tersedia dibagi dengan burn rate. Ini menunjukkan berapa lama perusahaan dapat bertahan sebelum kehabisan uang. Misalnya, jika Anda memiliki \$210,000 dan burn rate \$30,000, maka runway Anda adalah 7 bulan.
    \item Runway membantu pengusaha merencanakan kebutuhan pendanaan di masa depan.
\end{itemize}

\subsection{Fume Date}
\begin{itemize}
    \item \textbf{Fume Date} adalah tanggal di mana perusahaan tidak memiliki runway lagi, atau saat saldo kas mencapai nol. Ini adalah titik kritis yang harus diperhatikan oleh pengusaha.
    \item Fume date menunjukkan kapan perusahaan harus mencari pendanaan tambahan atau mengurangi pengeluaran untuk bertahan.
\end{itemize}

\subsection{Perencanaan Arus Kas}
\begin{itemize}
    \item Pengusaha sering melakukan perencanaan arus kas yang lebih rinci untuk memperkirakan burn rate dan fume date, terutama jika ada variabilitas dalam pengeluaran atau pendapatan.
    \item Grafik proyeksi arus kas dapat membantu memperkirakan kapan perusahaan akan kehabisan uang, memberikan gambaran yang lebih jelas tentang kesehatan finansial perusahaan.
\end{itemize}

\subsection{Breakeven Time (Payback Time)}
\begin{itemize}
    \item Breakeven time atau payback time adalah waktu yang diperlukan untuk mengembalikan investasi awal dari arus kas positif.
    \item Contoh: Jika Anda menginvestasikan \$40,000 untuk truk dan menghemat \$1,000 per bulan, maka payback period adalah 40 bulan (40,000 dibagi 1,000).
\end{itemize}

\subsection{Breakeven Quantity}
\begin{itemize}
    \item Breakeven quantity adalah jumlah unit yang harus dijual untuk menutupi biaya tetap.
    \item  Rumus:
          \begin{align*}
              Q = F / (P - C) .
          \end{align*}
          With:
          \begin{itemize}
              \item Q = Breakeven Quantity
              \item F = Fixed Cost
              \item P = Price
              \item C = Cost
          \end{itemize}
    \item Contoh: Jika biaya tetap tahunan adalah \$300,000, harga jual \$25, dan biaya produksi \$13.44, maka breakeven quantity adalah 25,952 unit per tahun atau sekitar 2,200 unit per bulan.
\end{itemize}

\subsection{Masalah dalam Analisis Breakeven}
\begin{itemize}
    \item Analisis breakeven berguna untuk menguji intuisi tentang seberapa banyak jalan bisnis yang diperlukan untuk bertahan.
    \item Namun, analisis ini mengasumsikan bahwa biaya tetap (F) adalah tetap, padahal biaya tersebut dapat berubah berdasarkan keputusan manajerial.
\end{itemize}

\newpage

\section{Elements of the Pitch and Exit Strategies}
\subsection{Pitch Deck dan Executive Summary}
\begin{itemize}
    \item \textbf{Pitch Deck}: Sebuah presentasi yang terdiri dari 10-12 slide yang menjelaskan ide bisnis Anda kepada investor. Ini biasanya dikirim melalui email setelah pertemuan awal.
    \item \textbf{Executive Summary}: Dokumen dua halaman yang merangkum informasi dari pitch deck dan menjelaskan ide bisnis dengan jelas.
\end{itemize}

\subsection{Struktur Pitch Deck}
\begin{itemize}
    \item Overview atau Hook Slide:
          \begin{itemize}
              \item Menyajikan ringkasan ide bisnis atau menarik perhatian dengan cerita atau fakta menarik.
              \item Hook: Elemen yang menarik perhatian audiens, seperti statistik atau cerita pribadi.
          \end{itemize}
    \item Problem Slide:
          \begin{itemize}
              \item Menjelaskan masalah yang ingin dipecahkan oleh produk Anda.
              \item Problem: Tantangan atau kesulitan yang dihadapi oleh pelanggan yang produk Anda dapat selesaikan.
          \end{itemize}
    \item Solution Slide:
          \begin{itemize}
              \item Memperkenalkan produk dan bagaimana produk tersebut menyelesaikan masalah.
              \item Solution: Produk atau layanan yang ditawarkan untuk mengatasi masalah yang telah diidentifikasi.
          \end{itemize}
    \item Technology atau Magic Slide:
          \begin{itemize}
              \item Menunjukkan keunggulan kompetitif produk Anda.
              \item Competitive Advantage: Faktor yang membuat produk Anda lebih baik dibandingkan dengan pesaing.
          \end{itemize}
    \item Market Sizing Slide:
          \begin{itemize}
              \item Menunjukkan ukuran pasar yang dapat dijangkau oleh produk Anda.
              \item Market Size: Estimasi total potensi pendapatan dari pasar yang dapat diakses.
          \end{itemize}
    \item Competition Slide:
          \begin{itemize}
              \item Menyajikan analisis pesaing dan bagaimana produk Anda bersaing.
              \item Competition: Perusahaan lain yang menawarkan produk serupa di pasar.
          \end{itemize}
    \item Go-to-Market Slide:
          \begin{itemize}
              \item Menjelaskan strategi akuisisi pelanggan.
              \item Customer Acquisition: Proses mendapatkan pelanggan baru untuk produk atau layanan Anda.
          \end{itemize}
    \item Team Slide:
          \begin{itemize}
              \item Menampilkan anggota tim dan kualifikasi mereka.
              \item Team: Individu yang terlibat dalam menjalankan bisnis dan memiliki keahlian yang relevan.
          \end{itemize}
    \item Projections Slide:
          \begin{itemize}
              \item Menyajikan proyeksi keuangan dan pertumbuhan bisnis.
              \item Financial Projections: Estimasi pendapatan dan pengeluaran di masa depan.
          \end{itemize}
    \item Roadmap, Milestones, dan Asks Slide:
          \begin{itemize}
              \item Menunjukkan rencana masa depan dan apa yang Anda butuhkan dari investor.
              \item Milestones: Pencapaian penting yang ingin diraih dalam waktu tertentu.
          \end{itemize}
\end{itemize}

\subsection{Tips untuk Pitching}
\begin{itemize}
    \item Kenali audiens Anda dan sesuaikan presentasi dengan tingkat pemahaman mereka.
    \item Gunakan demo produk untuk menunjukkan nilai nyata dari produk Anda.
    \item Siapkan diri untuk pertanyaan dan diskusi yang mungkin muncul selama presentasi.
\end{itemize}

\subsection{Pengertian Pro Forma Financial Statement}
\begin{itemize}
    \item Pro Forma: Istilah Latin yang berarti "sebagai bentuk". Ini merujuk pada laporan keuangan yang dipersiapkan untuk memperkirakan kinerja masa depan, bukan laporan akuntansi aktual.
    \item Berfungsi sebagai alat perencanaan untuk masa depan yang belum terjadi.
    \item Ada 3 skenario: Realistik, Konservatif, dan Optimistik
    \item Pro forma financial statements adalah alat penting untuk merencanakan masa depan bisnis Anda. Dengan menyusun skenario konservatif, realistis, dan optimis, Anda dapat memberikan gambaran yang lebih komprehensif kepada investor tentang potensi bisnis Anda dan bagaimana Anda berencana untuk menghadapinya.
\end{itemize}

\subsection{Jenis Laporan Keuangan Pro Forma}
\begin{itemize}
    \item Income Statement: Laporan yang menunjukkan pendapatan dan biaya untuk menghitung laba.
    \item Cash Flow Statement: Laporan yang menunjukkan aliran kas masuk dan keluar dari bisnis.
    \item Balance Sheet: Laporan yang menunjukkan aset, kewajiban, dan ekuitas pemegang saham pada suatu waktu tertentu.
\end{itemize}

\subsection{Analisis Pro Forma}
\begin{itemize}
    \item Proyeksi biasanya dilakukan secara bulanan untuk periode 36 bulan (3 tahun).
          Penting untuk menyertakan budget (anggaran) 12 bulan yang merinci rencana pengeluaran.
\end{itemize}

\subsection{Pertanyaan yang Dijawab oleh Pro Forma Financial Statements}
\begin{enumerate}
    \item Apa yang saya percayai tentang kinerja keuangan bisnis di masa depan?
          \begin{itemize}
              \item Menunjukkan potensi pendapatan dan laba di tahun-tahun mendatang.
          \end{itemize}
    \item Apa hubungan antara indikator kinerja utama dan kinerja keuangan?
          \begin{itemize}
              \item Menghubungkan faktor-faktor yang dapat dikendalikan manajer dengan hasil keuangan.
          \end{itemize}
    \item Berapa banyak kas yang saya butuhkan sebelum mencapai breakeven?
          \begin{itemize}
              \item Menghitung kebutuhan kas untuk menutupi arus kas negatif di awal bisnis.

          \end{itemize}
    \item Apakah rencana operasional saya konsisten dengan rencana keuangan?
          \begin{itemize}
              \item Memastikan bahwa rencana tindakan sejalan dengan proyeksi keuangan.

          \end{itemize}
\end{enumerate}

\subsection{Contoh Praktis Laporan Pro Forma}
\begin{itemize}
    \item Dalam contoh Carl's Openers, yang memproduksi pembuka botol dan pembuka anggur, laporan keuangan dimulai dengan Income Statement.
    \item Pendapatan dihitung berdasarkan parameter seperti jumlah pengecer dan rata-rata penjualan per pengecer.
\end{itemize}

\subsection{Praktik Terbaik untuk Pro Forma}
\begin{itemize}
    \item Bangun model menggunakan key revenue and cost drivers (penggerak pendapatan dan biaya utama) untuk kejelasan.
    \item Mulai dengan Income Statement dan gunakan untuk menggerakkan Cash Flow Statement dan Balance Sheet.
    \item Model keuangan biasanya disusun bulanan untuk 36 bulan dengan ringkasan tahunan.
\end{itemize}

\subsection{Definisi Entrepreneural Exit}
\begin{itemize}
    \item Entrepreneurial Exit: Suatu peristiwa likuiditas bagi pemangku kepentingan perusahaan, seperti pendiri dan investor, yang biasanya melibatkan pembelian kembali saham mereka.
    \item Liquidity Event: Peristiwa yang memungkinkan pemegang saham untuk menjual saham mereka dan mendapatkan uang tunai.
\end{itemize}

\subsection{Jenis-jenis Exit}
\begin{itemize}
    \item Acquisition (Akuisisi):
          \begin{itemize}
              \item Pro: Dapat memberikan organizational synergy, di mana perusahaan yang diakuisisi mendapatkan akses ke pasar yang lebih besar dan sumber daya yang lebih cepat.
              \item Kontra: Pendiri mungkin kehilangan kontrol atas perusahaan setelah akuisisi.
          \end{itemize}
    \item Initial Public Offering (IPO):
          \begin{itemize}
              \item Pro: Meningkatkan visibilitas dan kredibilitas perusahaan, serta dapat digunakan sebagai mata uang untuk akuisisi lainnya.
              \item Kontra: Biaya tinggi dan tekanan untuk memenuhi ekspektasi pasar, yang dapat mengalihkan fokus dari pengembangan produk.
          \end{itemize}
\end{itemize}

\subsection{Proses IPO di AS}
\begin{itemize}
    \item Underwriter: Bank investasi yang membantu memasarkan saham perusahaan kepada investor dan mengatur proses IPO.
    \item Underpricing: Selisih antara harga saham saat mulai diperdagangkan dan harga penutupan pada hari yang sama, yang sering kali merugikan perusahaan karena keuntungan tersebut tidak diterima oleh perusahaan.
\end{itemize}

\subsection{Penelitian tentang Exit dan Inovasi}
\begin{itemize}
    \item Penelitian menunjukkan bahwa private ownership (kepemilikan pribadi) menghasilkan hasil inovasi terbaik, sedangkan public ownership (kepemilikan publik) menghasilkan hasil terburuk. Hal ini disebabkan oleh tekanan untuk menghasilkan hasil jangka pendek di pasar publik.
\end{itemize}

\subsection{Kesimpulan terkait Entrepreneural Exit}
\begin{itemize}
    \item Memikirkan tentang likuiditas sangat penting bagi pemegang saham, dan bentuk exit yang dipilih dapat mempengaruhi strategi dan inovasi perusahaan di masa depan.
\end{itemize}

\end{document}