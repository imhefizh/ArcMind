% section | subsection | subsubsection | paragraph | subparagraph | verbatim | enumerate | item

% Just let it be an Article bro
\documentclass{article}

% Cover Metadata
\title{Growth Strategies}
\date{10 Oktober 2025}
\author{Maulana Hafidz Ismail}

% Image Setup
\usepackage{graphicx}
\usepackage{float}

% Link Setup
\usepackage{hyperref}

% Highlighting Setup
\usepackage{soul}
\usepackage{xcolor}
\sethlcolor{red!30}
\newcommand{\hlblue}[1]{\sethlcolor{cyan!30}\hl{#1}\sethlcolor{yellow}}

% Line Width
\sloppy
\setlength{\emergencystretch}{3em}
\hbadness=99999

% Code Syntax Setup
\usepackage{listings}
\usepackage{xcolor} % untuk warna
\lstset{
basicstyle=\ttfamily\small,
frame=single,
breaklines=true,
backgroundcolor=\color{gray!10},
keywordstyle=\color{blue},
commentstyle=\color{gray},
stringstyle=\color{red},
tabsize=2,
captionpos=b
}

% Start of the Article
\begin{document}

% Cover Page
\pagenumbering{gobble}
\maketitle
\newpage

% Table of Content Page
\tableofcontents
\newpage
\pagenumbering{arabic}

% Main Content
\section{Acquiring Customers and Forecasting Demand}
\subsection{KEY QUESTION}
\begin{enumerate}
    \item How do I forecast demand?
    \item How do I attract customers?
    \item How do I measure and report performance?
    \item  How do I build an organization for growth?
    \item  When should I rely on partners instead of doing something myself?
\end{enumerate}
\space

\subsection{Pengertian Two-Sided Markets}
\begin{itemize}
    \item Two-sided markets adalah pasar dengan dua jenis peserta, di mana manfaat untuk satu sisi tergantung pada jumlah peserta di sisi lainnya.
    \item Contoh: Jaringan kartu kredit seperti Visa dan MasterCard, di mana nilai bergabung untuk konsumen tergantung pada jumlah bisnis yang menerima kartu tersebut.
    \item Two-Sided Markets juga dikenal dengan istilah \textbf{Cross-Side Network Effects}
\end{itemize}

\subsection{Contoh Cross-Side Network Effects}
\begin{itemize}
    \item Gaming Consoles: Nilai konsumen dari konsol seperti Xbox atau PlayStation tergantung pada jumlah game yang tersedia, sedangkan nilai bagi pengembang game tergantung pada jumlah pengguna konsol tersebut.
    \item Uber: Nilai bagi konsumen menggunakan Uber tergantung pada jumlah taksi yang tersedia, dan nilai bagi pengemudi taksi tergantung pada jumlah konsumen yang menggunakan Uber.
\end{itemize}

\subsection{Memahami Tangan Likuiditas}
\begin{itemize}
    \item \textbf{Tangan likuiditas} mengacu pada situasi di mana ada kesulitan dalam menarik cukup banyak pengguna atau penyedia untuk menciptakan nilai bagi kedua belah pihak.
    \item Misalnya, jika sebuah platform seperti Uber tidak memiliki cukup pengemudi, maka pengguna tidak akan mendapatkan layanan yang mereka butuhkan, dan sebaliknya, jika tidak ada cukup pengguna, pengemudi tidak akan tertarik untuk bergabung.
    \item Istilah "tangan likuiditas" dalam konteks bisnis dan ekonomi biasanya merujuk pada tantangan yang dihadapi oleh perusahaan atau platform yang beroperasi dalam two-sided markets (pasar dua sisi). Dalam pasar ini, keberhasilan satu sisi (misalnya, pengguna) bergantung pada keberadaan dan partisipasi sisi lainnya (misalnya, penyedia layanan).
\end{itemize}

\subsection{Strategi untuk Mengatasi Tangan Likuiditas}
\begin{itemize}
    \item Pricing Strategy: Mengatur harga untuk memaksimalkan pendapatan dari kedua sisi pasar. Namun, strategi yang lebih efektif adalah memberikan subsidi kepada satu sisi pasar untuk meningkatkan pendapatan dari sisi lainnya. Contoh: Adobe Acrobat Reader yang gratis untuk konsumen, sementara Acrobat Writer dikenakan biaya untuk penerbit.
    \item Mengambil pasokan dari pasar lain, seperti yang dilakukan AirBnB dengan mengumpulkan listing dari Craigslist.
    \item Memanfaatkan influencer untuk menarik pengguna ke platform.
    \item Melakukan pencocokan manual antara penawaran dan permintaan jika transaksi tidak memerlukan pemenuhan waktu nyata.
\end{itemize}

\subsection{Proses Diffusi dan Adopsi}
\begin{itemize}
    \item \textbf{Difusi} adalah proses di mana suatu inovasi, produk, atau ide baru menyebar di antara individu atau kelompok dalam suatu masyarakat. Proses ini mencakup bagaimana informasi tentang inovasi tersebut disebarkan dan diterima oleh orang-orang, serta seberapa cepat dan luas inovasi tersebut diadopsi oleh pasar. Difusi sering kali dipengaruhi oleh berbagai faktor, termasuk karakteristik produk, saluran komunikasi, dan konteks sosial.
    \item  A\textbf{dopsi} adalah tindakan di mana individu atau kelompok mulai menggunakan atau mengintegrasikan inovasi, produk, atau ide baru ke dalam kehidupan mereka. Proses adopsi melibatkan keputusan untuk menerima dan menggunakan inovasi tersebut, yang dapat dipengaruhi oleh faktor-faktor seperti manfaat yang dirasakan, kemudahan penggunaan, dan kesesuaian dengan kebutuhan atau gaya hidup pengguna. Adopsi dapat dibagi menjadi beberapa tahap, mulai dari pengenalan hingga penggunaan penuh.
    \item Diffusi dan adopsi produk baru seringkali berlangsung lambat, meskipun produk tersebut menarik.
    \item Contoh yang diberikan adalah EZPass, yang menunjukkan bahwa meskipun ada teknologi yang bermanfaat, tidak semua orang mengadopsinya dengan cepat
\end{itemize}

\subsection{Parameter Kunci dalam Adopsi}
\begin{itemize}
    \item Dua parameter utama dalam adopsi adalah:
          \begin{enumerate}
              \item \textbf{Time to Take Off}: Waktu yang diperlukan dari pengenalan produk hingga adopsi yang cepat.
              \item \textbf{Peak Adoption Rate}: Kecepatan pertumbuhan adopsi setelah produk mulai diadopsi.
          \end{enumerate}
\end{itemize}

\subsection{Kategorisasi Adopter oleh Everett Rogers}
\begin{itemize}
    \item Everett Rogers mengidentifikasi lima kategori adopter:
          \begin{enumerate}
              \item \textbf{Innovators}: Pengadopsi awal.
              \item \textbf{Early Adopters}: Mereka yang mengadopsi setelah inovator.
              \item \textbf{Early Majority}: Mayoritas awal yang mengadopsi setelah early adopters.
              \item \textbf{Late Majority}: Mayoritas yang mengadopsi setelah early majority.
              \item \textbf{Laggards}: Mereka yang mengadopsi paling akhir.
          \end{enumerate}
\end{itemize}

\subsection{Faktor yang Mempengaruhi Kecepatan Diffusi}
\begin{itemize}
    \item Rogers mengidentifikasi lima faktor intrinsik yang mempengaruhi kecepatan diffusi:
          \begin{enumerate}
              \item \textbf{Relative Advantage}: Seberapa baik produk baru dibandingkan alternatif yang ada.
              \item \textbf{Visibility}: Seberapa terlihat produk tersebut saat digunakan oleh orang lain.
              \item \textbf{Trialability}: Kemudahan untuk mencoba produk sebelum membeli.
              \item \textbf{Simplicity}: Seberapa mudah produk dipahami dan digunakan.
              \item \textbf{Compatibility}: Seberapa baik produk cocok dengan kehidupan pengguna.
          \end{enumerate}
\end{itemize}

\subsection{Kesimpulan terkait Difusi dan Adopsi}
\begin{itemize}
    \item Waktu untuk take off biasanya lebih lama dari yang diharapkan, sering kali memakan waktu dua hingga lima tahun untuk inovasi baru.
    \item Beberapa atribut produk dapat dimodifikasi untuk meningkatkan adopsi, seperti trialability.
    \item Penting untuk realistis dalam merencanakan waktu dan kecepatan adopsi produk baru.
\end{itemize}

\subsection{Tujuan dan Pentingnya Forecasting}
\begin{itemize}
    \item Memahami bahwa baik under forecasting maupun over forecasting dapat merugikan. Under forecasting dapat menyebabkan kehilangan peluang.
    \item Forecasting bukanlah hal yang mudah, tetapi ada metode sederhana yang dapat digunakan untuk meningkatkan estimasi dan mempengaruhi permintaan.
\end{itemize}

\subsection{Model ACCORD oleh Prof. Everett Rogers}
\begin{itemize}
    \item \textbf{A (Advantage)}: Keuntungan relatif dari ide, baik dalam hal ekonomi maupun manfaat emosional.
    \item \textbf{C (Compatibility)}: Seberapa mirip metode yang diperlukan untuk menggunakan ide baru dengan produk yang sudah ada.
    \item \textbf{C (Complexity)}: Tingkat kemudahan dalam memahami bagaimana ide baru akan berfungsi.
    \item \textbf{O (Observability)}: Apakah pengguna lain dapat melihat adopsi ide ini dan menirunya.
    \item \textbf{R (Risk)}: Risiko yang terkait dengan adopsi ide baru, baik dari segi ekonomi maupun sosial.
    \item \textbf{D (Divisibility)}: Kemampuan untuk mencoba ide baru dalam skala kecil sebelum mengadopsi sepenuhnya.
\end{itemize}

\subsection{Contoh Penerapan Model ACCORD}
\begin{itemize}
    \item Advantage (Keuntungan): Nilai dari 1 (tidak ada keuntungan) hingga 7 (keuntungan yang sangat signifikan). Pertimbangkan seberapa besar manfaat yang akan didapatkan oleh pengguna dari ide Anda.
    \item Compatibility (Kesesuaian): Nilai dari 1 (sangat tidak kompatibel) hingga 7 (sangat kompatibel). Tanyakan pada diri sendiri seberapa mudah ide Anda dapat diterima dan digunakan oleh pengguna yang sudah ada.
    \item Complexity (Kompleksitas): Nilai dari 1 (sangat kompleks) hingga 7 (sangat sederhana). Evaluasi seberapa mudah orang dapat memahami dan menggunakan ide Anda.
    \item Observability (Observabilitas): Nilai dari 1 (tidak terlihat) hingga 7 (sangat terlihat). Pertimbangkan apakah orang lain dapat melihat dan meniru penggunaan ide Anda.
    \item Risk (Risiko): Nilai dari 1 (risiko sangat tinggi) hingga 7 (risiko sangat rendah). Tanyakan seberapa besar risiko yang dihadapi pengguna jika mereka mengadopsi ide Anda.
    \item Divisibility (Divisibilitas): Nilai dari 1 (tidak dapat dicoba) hingga 7 (dapat dicoba dengan mudah). Pertimbangkan apakah pengguna dapat mencoba ide Anda dalam skala kecil sebelum mengadopsi sepenuhnya.
\end{itemize}

\subsection{Berbicara dengan Para Ahli}
\begin{itemize}
    \item Jika Anda meluncurkan produk baru, seperti medical device, penting untuk berbicara dengan dokter, perawat, dan administrasi rumah sakit.
    \item Untuk konsep retail baru, ajak bicara calon pembeli, toko retail, dan pemasok.
\end{itemize}

\subsection{Forecasting Demand}
\begin{itemize}
    \item Mintalah para ahli memberikan prediksi mereka secara kuantitatif dan tanyakan alasan di balik prediksi tersebut.
    \item Menggabungkan berbagai prediksi dapat mengurangi kesalahan, karena beberapa mungkin terlalu optimis dan yang lain pesimis.
\end{itemize}

\subsection{Analogi dan Ide Lain}
\begin{itemize}
    \item Periksa produk serupa di kategori lain untuk memahami keberhasilan mereka.
    \item Misalnya, jika meluncurkan direct to consumer pet food, lihat apa yang dibeli oleh pembeli serupa baru-baru ini.
\end{itemize}

\subsection{Sumber Tambahan terkait Bacaan}
\begin{itemize}
    \item Disarankan untuk membaca karya Professor Scott Armstrong dan Professor Phil Tetlock untuk pemahaman lebih dalam tentang forecasting.
\end{itemize}

\subsection{Contoh Penerapan Demand Decomposition untuk Car-Sharing di Philadelphia}
\begin{itemize}
    \item Dimulai dengan populasi Philadelphia untuk menentukan jumlah orang yang membutuhkan transportasi.
    \item Menghitung fraksi/pecahan orang yang lebih memilih transportasi pribadi dibandingkan transportasi umum, serta yang tidak memiliki mobil.
    \item Mengalikan semua fraksi ini untuk mendapatkan estimasi permintaan untuk Uber atau car-sharing di Philadelphia.
\end{itemize}

\subsection{Contoh Penerapan Demand Decomposition untuk Toothbrush di India}
\begin{itemize}
    \item Memulai dengan populasi India dan mencari tahu berapa banyak orang yang menyikat gigi dengan alat tertentu.
    \item Mengumpulkan data tentang frekuensi menyikat gigi dan berapa kali sikat gigi diganti dalam setahun.
    \item Mengalikan semua fraksi ini untuk mendapatkan estimasi permintaan untuk toothbrush di India.
\end{itemize}

\subsection{Peningkatan Permintaan}
\begin{itemize}
    \item Demand decomposition tidak hanya memberikan estimasi permintaan, tetapi juga mengidentifikasi cara untuk meningkatkan permintaan.
    \item Strategi untuk meningkatkan permintaan termasuk mengedukasi masyarakat tentang pentingnya perawatan gigi dan mendorong penggunaan toothbrush dibandingkan dengan neem twigs.
\end{itemize}

\subsection{Kesimpulan Terkait Forecasting}
\begin{itemize}
    \item Meramalkan permintaan adalah aspek penting dalam meluncurkan usaha baru.
    \item Metode yang dibahas termasuk: \textbf{ACCORD model, metode kualitatif, dan demand decomposition}.
\end{itemize}

\section{Marketing and PR}
\subsection{Strategi Digital Marketing}
\begin{itemize}
    \item Mengklasifikasikan strategi tersebut ke dalam tiga kategori utama:
          \begin{itemize}
              \item Owned media
              \item Paid media
              \item Earned media.
          \end{itemize}
\end{itemize}

\subsection{Owned Media}
\begin{itemize}
    \item Owned media mencakup properti web yang Anda miliki atau buat, seperti website, aplikasi mobile, saluran media sosial, dan blog.
    \item Penting untuk merancang website agar mudah ditemukan oleh search engine dan untuk mempertimbangkan platform media sosial yang tepat untuk berinteraksi dengan audiens.
\end{itemize}

\subsection{Paid Media}
\begin{itemize}
    \item Paid media mencakup berbagai bentuk iklan, termasuk iklan di search engine, social ads, dan display ads.
    \item Affiliate marketing juga merupakan bagian dari paid media, di mana Anda membayar mitra untuk mempromosikan produk Anda.
\end{itemize}

\subsection{Earned Media}
\begin{itemize}
    \item Earned media adalah sebutan untuk penyebutan produk Anda oleh pihak ketiga, seperti blogger atau pelanggan yang membahas produk Anda di media sosial.
    \item Strategi untuk meningkatkan earned media termasuk menawarkan insentif kepada pelanggan untuk membahas produk Anda dan mengidentifikasi serta melibatkan influencer.
\end{itemize}

\subsection{Pentingnya Social Media untuk Pemasar}
\begin{itemize}
    \item Social media memberikan jangkauan yang sangat besar, dengan platform seperti Facebook yang memiliki hampir 1,5 miliar pengguna.
    \item Sekitar 28\%\ waktu yang dihabiskan pengguna online dihabiskan di social networking sites, menjadikannya medium yang menarik bagi pemasar.
\end{itemize}

\subsection{Contoh Sukses dalam Pemasaran Social Media}
\begin{itemize}
    \item Blendtec menggunakan video di YouTube untuk menunjukkan produk mereka, yang meningkatkan penjualan hingga hampir 400%.
    \item Runkeeper memanfaatkan Facebook untuk berinteraksi dengan konsumen dan membangun pengikut yang besar.
\end{itemize}

\subsection{Pengaruh Social Cues dalam Pemasaran}
\begin{itemize}
    \item Eksperimen oleh Professor Matt Salganik menunjukkan bahwa social cues dapat membuat pasar menjadi lebih hits-driven dan tidak terduga.
    \item Produk yang dipilih oleh pengguna awal cenderung lebih populer di kalangan peserta berikutnya, menunjukkan kekuatan social cues dalam penemuan produk.
\end{itemize}

\subsection{Strategi Pemasaran Social Media}
\begin{itemize}
    \item Pemasaran social media mencakup kehadiran di platform seperti Facebook, Twitter, dan LinkedIn, serta blogging dan viral marketing.
    \item Penting untuk mengidentifikasi platform social media yang tepat berdasarkan di mana pelanggan berada.
\end{itemize}

\subsection{Tantangan dalam Pemasaran Social Media}
\begin{itemize}
    \item Hanya 0,2\%\ dari status update di Facebook yang mencapai pengguna akhir, sehingga sulit untuk mendapatkan perhatian di tengah banyaknya konten.
    \item Konten yang dirancang dengan baik, seperti yang emosional atau humoris, cenderung mendapatkan lebih banyak keterlibatan.
\end{itemize}

\subsection{mengintegrasikan Social Media dalam Customer Journey}
\begin{itemize}
    \item Contoh Warby Parker menunjukkan bagaimana mereka menggunakan social media untuk memenuhi kebutuhan sosial pelanggan saat memilih produk.
    \item GoPro memanfaatkan YouTube untuk membantu pelanggan berbagi video setelah pembelian, menunjukkan relevansi social media pasca pembelian.
\end{itemize}

\subsection{Fungsi Utama Mesin Pencari}
\begin{itemize}
    \item \textbf{Crawling}: Mesin pencari mengumpulkan semua halaman di web.
    \item \textbf{Indexing}: Mengorganisir konten halaman yang telah diunduh untuk memudahkan pencarian di masa mendatang.
    \item \textbf{Ranking}: Menentukan relevansi halaman terhadap query pencarian dan mengurutkannya berdasarkan relevansi.
\end{itemize}

\subsection{Strategi Penting untuk SEO}
\begin{itemize}
    \item \textbf{Crawling}: Pastikan semua halaman web Anda dapat di-crawl oleh mesin pencari. Gunakan sitemap untuk memudahkan mesin pencari menemukan semua halaman.
    \item \textbf{Indexing}: Gunakan keywords yang relevan di halaman web Anda untuk memastikan halaman Anda terindeks dengan baik. Pilih keywords berdasarkan volume pencarian dan tingkat persaingan.
    \item \textbf{Ranking}: Peringkat di mesin pencari dipengaruhi oleh penggunaan keywords dan jumlah inlinks. Dapatkan banyak inlinks dari situs berkualitas tinggi untuk meningkatkan otoritas halaman Anda.
\end{itemize}

\subsection{Peran Blog dalam SEO}
\begin{itemize}
    \item Blog dapat meningkatkan jumlah konten di situs Anda, yang membantu dalam proses indexing dan menarik lebih banyak traffic.
\end{itemize}

\subsection{Tujuan Kampanye Iklan Online}
\begin{itemize}
    \item Tujuan kampanye iklan online dapat bervariasi, seperti akuisisi pelanggan jangka panjang, mendukung peluncuran produk, atau mencocokkan biaya per akuisisi dengan saluran iklan lainnya.
    \item Penting untuk memahami bahwa meskipun digital advertising lebih terarah dibandingkan dengan iklan offline, biaya dapat tinggi dan tidak selalu menguntungkan.
\end{itemize}

\subsection{Opsi dalam Digital Advertising}
\begin{itemize}
    \item \textbf{Search Engine Marketing}: Memposting iklan di mesin pencari seperti Google, dengan tingkat niat beli yang tinggi.
    \item \textbf{Display Ads}: Iklan gambar atau video di situs web premium atau blog, dengan jangkauan yang luas tetapi variabilitas dalam kualitas dan kinerja.
    \item \textbf{Social Ads}: Iklan di platform media sosial seperti Facebook dan Twitter, dengan banyak opsi penargetan tetapi mungkin sulit untuk mengukur manfaatnya.
\end{itemize}

\subsection{Pro dan Kontra dari Setiap Opsi Digital Ad}
\begin{itemize}
    \item Search Engine Advertising:
          \begin{itemize}
              \item Pro: Tingkat respons yang tinggi dan format yang sederhana.
              \item Kontra: Biaya yang tinggi dan informasi demografis yang terbatas.
          \end{itemize}
    \item Display Ads:
          \begin{itemize}
              \item Pro: Jangkauan yang luas dan berbagai format.
              \item Kontra: Kualitas dan kinerja yang bervariasi, serta rendahnya minat pelanggan.
          \end{itemize}
    \item Social Ads:
          \begin{itemize}
              \item Pro: Opsi penargetan yang kaya dan biaya yang lebih rendah.
              \item Kontra: Sulit untuk mengukur manfaat, terutama dalam konteks e-commerce.
          \end{itemize}
\end{itemize}

\subsection{Langkah-langkah dalam Search Engine Marketing}
\begin{enumerate}
    \item \textbf{Keyword Selection}: Mengidentifikasi kata kunci yang relevan untuk bisnis.
    \item \textbf{Ad Design}: Mendesain iklan yang jelas dan menarik.
    \item \textbf{Bidding}: Menentukan tawaran untuk setiap klik pada iklan.
\end{enumerate}

\subsection{Metrik untuk Mengukur Kinerja}
\begin{itemize}
    \item \textbf{Quantity of Exposure}: Jumlah tayangan iklan per dolar yang dibelanjakan.
    \item \textbf{Quality of Exposure}: Tingkat konversi dari klik menjadi hasil yang diinginkan.
    \item \textbf{Cost of Exposure}: Mengukur biaya per klik untuk mengevaluasi efektivitas iklan.
\end{itemize}

\subsection{Kesimpulan terkait Digital Advertising}
\begin{itemize}
    \item Digital advertising adalah aktivitas yang didorong oleh data, memerlukan eksperimen yang konstan dan fokus pada pengukuran untuk mengoptimalkan kampanye iklan. Keberhasilan dalam digital advertising bergantung pada kemampuan untuk memanfaatkan pengukuran dan penargetan secara efektif.
\end{itemize}

% END Of DOC
\end{document}