% section | subsection | subsubsection | paragraph | subparagraph | verbatim | enumerate | item

% Just let it be an Article bro
\documentclass{article}

% Cover Metadata
\title{Growth Strategies}
\date{10 Oktober 2025}
\author{Maulana Hafidz Ismail}

% Image Setup
\usepackage{graphicx}
\usepackage{float}

\usepackage{amsmath}

% Link Setup
\usepackage{hyperref}

% Highlighting Setup
\usepackage{soul}
\usepackage{xcolor}
\sethlcolor{red!30}
\newcommand{\hlblue}[1]{\sethlcolor{cyan!30}\hl{#1}\sethlcolor{yellow}}

% Line Width
\sloppy
\setlength{\emergencystretch}{3em}
\hbadness=99999

% Code Syntax Setup
\usepackage{listings}
\usepackage{xcolor} % untuk warna
\lstset{
basicstyle=\ttfamily\small,
frame=single,
breaklines=true,
backgroundcolor=\color{gray!10},
keywordstyle=\color{blue},
commentstyle=\color{gray},
stringstyle=\color{red},
tabsize=2,
captionpos=b
}

% Start of the Article
\begin{document}

% Cover Page
\pagenumbering{gobble}
\maketitle
\newpage

% Table of Content Page
\tableofcontents
\newpage
\pagenumbering{arabic}

% Main Content
\section{Acquiring Customers and Forecasting Demand}
\subsection{KEY QUESTION}
\begin{enumerate}
    \item How do I forecast demand?
    \item How do I attract customers?
    \item How do I measure and report performance?
    \item  How do I build an organization for growth?
    \item  When should I rely on partners instead of doing something myself?
\end{enumerate}
\space

\subsection{Pengertian Two-Sided Markets}
\begin{itemize}
    \item Two-sided markets adalah pasar dengan dua jenis peserta, di mana manfaat untuk satu sisi tergantung pada jumlah peserta di sisi lainnya.
    \item Contoh: Jaringan kartu kredit seperti Visa dan MasterCard, di mana nilai bergabung untuk konsumen tergantung pada jumlah bisnis yang menerima kartu tersebut.
    \item Two-Sided Markets juga dikenal dengan istilah \textbf{Cross-Side Network Effects}
\end{itemize}

\subsection{Contoh Cross-Side Network Effects}
\begin{itemize}
    \item Gaming Consoles: Nilai konsumen dari konsol seperti Xbox atau PlayStation tergantung pada jumlah game yang tersedia, sedangkan nilai bagi pengembang game tergantung pada jumlah pengguna konsol tersebut.
    \item Uber: Nilai bagi konsumen menggunakan Uber tergantung pada jumlah taksi yang tersedia, dan nilai bagi pengemudi taksi tergantung pada jumlah konsumen yang menggunakan Uber.
\end{itemize}

\subsection{Memahami Tangan Likuiditas}
\begin{itemize}
    \item \textbf{Tangan likuiditas} mengacu pada situasi di mana ada kesulitan dalam menarik cukup banyak pengguna atau penyedia untuk menciptakan nilai bagi kedua belah pihak.
    \item Misalnya, jika sebuah platform seperti Uber tidak memiliki cukup pengemudi, maka pengguna tidak akan mendapatkan layanan yang mereka butuhkan, dan sebaliknya, jika tidak ada cukup pengguna, pengemudi tidak akan tertarik untuk bergabung.
    \item Istilah "tangan likuiditas" dalam konteks bisnis dan ekonomi biasanya merujuk pada tantangan yang dihadapi oleh perusahaan atau platform yang beroperasi dalam two-sided markets (pasar dua sisi). Dalam pasar ini, keberhasilan satu sisi (misalnya, pengguna) bergantung pada keberadaan dan partisipasi sisi lainnya (misalnya, penyedia layanan).
\end{itemize}

\subsection{Strategi untuk Mengatasi Tangan Likuiditas}
\begin{itemize}
    \item Pricing Strategy: Mengatur harga untuk memaksimalkan pendapatan dari kedua sisi pasar. Namun, strategi yang lebih efektif adalah memberikan subsidi kepada satu sisi pasar untuk meningkatkan pendapatan dari sisi lainnya. Contoh: Adobe Acrobat Reader yang gratis untuk konsumen, sementara Acrobat Writer dikenakan biaya untuk penerbit.
    \item Mengambil pasokan dari pasar lain, seperti yang dilakukan AirBnB dengan mengumpulkan listing dari Craigslist.
    \item Memanfaatkan influencer untuk menarik pengguna ke platform.
    \item Melakukan pencocokan manual antara penawaran dan permintaan jika transaksi tidak memerlukan pemenuhan waktu nyata.
\end{itemize}

\subsection{Proses Diffusi dan Adopsi}
\begin{itemize}
    \item \textbf{Difusi} adalah proses di mana suatu inovasi, produk, atau ide baru menyebar di antara individu atau kelompok dalam suatu masyarakat. Proses ini mencakup bagaimana informasi tentang inovasi tersebut disebarkan dan diterima oleh orang-orang, serta seberapa cepat dan luas inovasi tersebut diadopsi oleh pasar. Difusi sering kali dipengaruhi oleh berbagai faktor, termasuk karakteristik produk, saluran komunikasi, dan konteks sosial.
    \item  A\textbf{dopsi} adalah tindakan di mana individu atau kelompok mulai menggunakan atau mengintegrasikan inovasi, produk, atau ide baru ke dalam kehidupan mereka. Proses adopsi melibatkan keputusan untuk menerima dan menggunakan inovasi tersebut, yang dapat dipengaruhi oleh faktor-faktor seperti manfaat yang dirasakan, kemudahan penggunaan, dan kesesuaian dengan kebutuhan atau gaya hidup pengguna. Adopsi dapat dibagi menjadi beberapa tahap, mulai dari pengenalan hingga penggunaan penuh.
    \item Diffusi dan adopsi produk baru seringkali berlangsung lambat, meskipun produk tersebut menarik.
    \item Contoh yang diberikan adalah EZPass, yang menunjukkan bahwa meskipun ada teknologi yang bermanfaat, tidak semua orang mengadopsinya dengan cepat
\end{itemize}

\subsection{Parameter Kunci dalam Adopsi}
\begin{itemize}
    \item Dua parameter utama dalam adopsi adalah:
          \begin{enumerate}
              \item \textbf{Time to Take Off}: Waktu yang diperlukan dari pengenalan produk hingga adopsi yang cepat.
              \item \textbf{Peak Adoption Rate}: Kecepatan pertumbuhan adopsi setelah produk mulai diadopsi.
          \end{enumerate}
\end{itemize}

\subsection{Kategorisasi Adopter oleh Everett Rogers}
\begin{itemize}
    \item Everett Rogers mengidentifikasi lima kategori adopter:
          \begin{enumerate}
              \item \textbf{Innovators}: Pengadopsi awal.
              \item \textbf{Early Adopters}: Mereka yang mengadopsi setelah inovator.
              \item \textbf{Early Majority}: Mayoritas awal yang mengadopsi setelah early adopters.
              \item \textbf{Late Majority}: Mayoritas yang mengadopsi setelah early majority.
              \item \textbf{Laggards}: Mereka yang mengadopsi paling akhir.
          \end{enumerate}
\end{itemize}

\subsection{Faktor yang Mempengaruhi Kecepatan Diffusi}
\begin{itemize}
    \item Rogers mengidentifikasi lima faktor intrinsik yang mempengaruhi kecepatan diffusi:
          \begin{enumerate}
              \item \textbf{Relative Advantage}: Seberapa baik produk baru dibandingkan alternatif yang ada.
              \item \textbf{Visibility}: Seberapa terlihat produk tersebut saat digunakan oleh orang lain.
              \item \textbf{Trialability}: Kemudahan untuk mencoba produk sebelum membeli.
              \item \textbf{Simplicity}: Seberapa mudah produk dipahami dan digunakan.
              \item \textbf{Compatibility}: Seberapa baik produk cocok dengan kehidupan pengguna.
          \end{enumerate}
\end{itemize}

\subsection{Kesimpulan terkait Difusi dan Adopsi}
\begin{itemize}
    \item Waktu untuk take off biasanya lebih lama dari yang diharapkan, sering kali memakan waktu dua hingga lima tahun untuk inovasi baru.
    \item Beberapa atribut produk dapat dimodifikasi untuk meningkatkan adopsi, seperti trialability.
    \item Penting untuk realistis dalam merencanakan waktu dan kecepatan adopsi produk baru.
\end{itemize}

\subsection{Tujuan dan Pentingnya Forecasting}
\begin{itemize}
    \item Memahami bahwa baik under forecasting maupun over forecasting dapat merugikan. Under forecasting dapat menyebabkan kehilangan peluang.
    \item Forecasting bukanlah hal yang mudah, tetapi ada metode sederhana yang dapat digunakan untuk meningkatkan estimasi dan mempengaruhi permintaan.
\end{itemize}

\subsection{Model ACCORD oleh Prof. Everett Rogers}
\begin{itemize}
    \item \textbf{A (Advantage)}: Keuntungan relatif dari ide, baik dalam hal ekonomi maupun manfaat emosional.
    \item \textbf{C (Compatibility)}: Seberapa mirip metode yang diperlukan untuk menggunakan ide baru dengan produk yang sudah ada.
    \item \textbf{C (Complexity)}: Tingkat kemudahan dalam memahami bagaimana ide baru akan berfungsi.
    \item \textbf{O (Observability)}: Apakah pengguna lain dapat melihat adopsi ide ini dan menirunya.
    \item \textbf{R (Risk)}: Risiko yang terkait dengan adopsi ide baru, baik dari segi ekonomi maupun sosial.
    \item \textbf{D (Divisibility)}: Kemampuan untuk mencoba ide baru dalam skala kecil sebelum mengadopsi sepenuhnya.
\end{itemize}

\subsection{Contoh Penerapan Model ACCORD}
\begin{itemize}
    \item Advantage (Keuntungan): Nilai dari 1 (tidak ada keuntungan) hingga 7 (keuntungan yang sangat signifikan). Pertimbangkan seberapa besar manfaat yang akan didapatkan oleh pengguna dari ide Anda.
    \item Compatibility (Kesesuaian): Nilai dari 1 (sangat tidak kompatibel) hingga 7 (sangat kompatibel). Tanyakan pada diri sendiri seberapa mudah ide Anda dapat diterima dan digunakan oleh pengguna yang sudah ada.
    \item Complexity (Kompleksitas): Nilai dari 1 (sangat kompleks) hingga 7 (sangat sederhana). Evaluasi seberapa mudah orang dapat memahami dan menggunakan ide Anda.
    \item Observability (Observabilitas): Nilai dari 1 (tidak terlihat) hingga 7 (sangat terlihat). Pertimbangkan apakah orang lain dapat melihat dan meniru penggunaan ide Anda.
    \item Risk (Risiko): Nilai dari 1 (risiko sangat tinggi) hingga 7 (risiko sangat rendah). Tanyakan seberapa besar risiko yang dihadapi pengguna jika mereka mengadopsi ide Anda.
    \item Divisibility (Divisibilitas): Nilai dari 1 (tidak dapat dicoba) hingga 7 (dapat dicoba dengan mudah). Pertimbangkan apakah pengguna dapat mencoba ide Anda dalam skala kecil sebelum mengadopsi sepenuhnya.
\end{itemize}

\subsection{Berbicara dengan Para Ahli}
\begin{itemize}
    \item Jika Anda meluncurkan produk baru, seperti medical device, penting untuk berbicara dengan dokter, perawat, dan administrasi rumah sakit.
    \item Untuk konsep retail baru, ajak bicara calon pembeli, toko retail, dan pemasok.
\end{itemize}

\subsection{Forecasting Demand}
\begin{itemize}
    \item Mintalah para ahli memberikan prediksi mereka secara kuantitatif dan tanyakan alasan di balik prediksi tersebut.
    \item Menggabungkan berbagai prediksi dapat mengurangi kesalahan, karena beberapa mungkin terlalu optimis dan yang lain pesimis.
\end{itemize}

\subsection{Analogi dan Ide Lain}
\begin{itemize}
    \item Periksa produk serupa di kategori lain untuk memahami keberhasilan mereka.
    \item Misalnya, jika meluncurkan direct to consumer pet food, lihat apa yang dibeli oleh pembeli serupa baru-baru ini.
\end{itemize}

\subsection{Sumber Tambahan terkait Bacaan}
\begin{itemize}
    \item Disarankan untuk membaca karya Professor Scott Armstrong dan Professor Phil Tetlock untuk pemahaman lebih dalam tentang forecasting.
\end{itemize}

\subsection{Contoh Penerapan Demand Decomposition untuk Car-Sharing di Philadelphia}
\begin{itemize}
    \item Dimulai dengan populasi Philadelphia untuk menentukan jumlah orang yang membutuhkan transportasi.
    \item Menghitung fraksi/pecahan orang yang lebih memilih transportasi pribadi dibandingkan transportasi umum, serta yang tidak memiliki mobil.
    \item Mengalikan semua fraksi ini untuk mendapatkan estimasi permintaan untuk Uber atau car-sharing di Philadelphia.
\end{itemize}

\subsection{Contoh Penerapan Demand Decomposition untuk Toothbrush di India}
\begin{itemize}
    \item Memulai dengan populasi India dan mencari tahu berapa banyak orang yang menyikat gigi dengan alat tertentu.
    \item Mengumpulkan data tentang frekuensi menyikat gigi dan berapa kali sikat gigi diganti dalam setahun.
    \item Mengalikan semua fraksi ini untuk mendapatkan estimasi permintaan untuk toothbrush di India.
\end{itemize}

\subsection{Peningkatan Permintaan}
\begin{itemize}
    \item Demand decomposition tidak hanya memberikan estimasi permintaan, tetapi juga mengidentifikasi cara untuk meningkatkan permintaan.
    \item Strategi untuk meningkatkan permintaan termasuk mengedukasi masyarakat tentang pentingnya perawatan gigi dan mendorong penggunaan toothbrush dibandingkan dengan neem twigs.
\end{itemize}

\subsection{Kesimpulan Terkait Forecasting}
\begin{itemize}
    \item Meramalkan permintaan adalah aspek penting dalam meluncurkan usaha baru.
    \item Metode yang dibahas termasuk: \textbf{ACCORD model, metode kualitatif, dan demand decomposition}.
\end{itemize}
\newpage
\section{Marketing and PR}
\subsection{Strategi Digital Marketing}
\begin{itemize}
    \item Mengklasifikasikan strategi tersebut ke dalam tiga kategori utama:
          \begin{itemize}
              \item Owned media
              \item Paid media
              \item Earned media.
          \end{itemize}
\end{itemize}

\subsection{Owned Media}
\begin{itemize}
    \item Owned media mencakup properti web yang Anda miliki atau buat, seperti website, aplikasi mobile, saluran media sosial, dan blog.
    \item Penting untuk merancang website agar mudah ditemukan oleh search engine dan untuk mempertimbangkan platform media sosial yang tepat untuk berinteraksi dengan audiens.
\end{itemize}

\subsection{Paid Media}
\begin{itemize}
    \item Paid media mencakup berbagai bentuk iklan, termasuk iklan di search engine, social ads, dan display ads.
    \item Affiliate marketing juga merupakan bagian dari paid media, di mana Anda membayar mitra untuk mempromosikan produk Anda.
\end{itemize}

\subsection{Earned Media}
\begin{itemize}
    \item Earned media adalah sebutan untuk penyebutan produk Anda oleh pihak ketiga, seperti blogger atau pelanggan yang membahas produk Anda di media sosial.
    \item Strategi untuk meningkatkan earned media termasuk menawarkan insentif kepada pelanggan untuk membahas produk Anda dan mengidentifikasi serta melibatkan influencer.
\end{itemize}

\subsection{Pentingnya Social Media untuk Pemasar}
\begin{itemize}
    \item Social media memberikan jangkauan yang sangat besar, dengan platform seperti Facebook yang memiliki hampir 1,5 miliar pengguna.
    \item Sekitar 28\%\ waktu yang dihabiskan pengguna online dihabiskan di social networking sites, menjadikannya medium yang menarik bagi pemasar.
\end{itemize}

\subsection{Contoh Sukses dalam Pemasaran Social Media}
\begin{itemize}
    \item Blendtec menggunakan video di YouTube untuk menunjukkan produk mereka, yang meningkatkan penjualan hingga hampir 400%.
    \item Runkeeper memanfaatkan Facebook untuk berinteraksi dengan konsumen dan membangun pengikut yang besar.
\end{itemize}

\subsection{Pengaruh Social Cues dalam Pemasaran}
\begin{itemize}
    \item Eksperimen oleh Professor Matt Salganik menunjukkan bahwa social cues dapat membuat pasar menjadi lebih hits-driven dan tidak terduga.
    \item Produk yang dipilih oleh pengguna awal cenderung lebih populer di kalangan peserta berikutnya, menunjukkan kekuatan social cues dalam penemuan produk.
\end{itemize}

\subsection{Strategi Pemasaran Social Media}
\begin{itemize}
    \item Pemasaran social media mencakup kehadiran di platform seperti Facebook, Twitter, dan LinkedIn, serta blogging dan viral marketing.
    \item Penting untuk mengidentifikasi platform social media yang tepat berdasarkan di mana pelanggan berada.
\end{itemize}

\subsection{Tantangan dalam Pemasaran Social Media}
\begin{itemize}
    \item Hanya 0,2\%\ dari status update di Facebook yang mencapai pengguna akhir, sehingga sulit untuk mendapatkan perhatian di tengah banyaknya konten.
    \item Konten yang dirancang dengan baik, seperti yang emosional atau humoris, cenderung mendapatkan lebih banyak keterlibatan.
\end{itemize}

\subsection{mengintegrasikan Social Media dalam Customer Journey}
\begin{itemize}
    \item Contoh Warby Parker menunjukkan bagaimana mereka menggunakan social media untuk memenuhi kebutuhan sosial pelanggan saat memilih produk.
    \item GoPro memanfaatkan YouTube untuk membantu pelanggan berbagi video setelah pembelian, menunjukkan relevansi social media pasca pembelian.
\end{itemize}

\subsection{Fungsi Utama Mesin Pencari}
\begin{itemize}
    \item \textbf{Crawling}: Mesin pencari mengumpulkan semua halaman di web.
    \item \textbf{Indexing}: Mengorganisir konten halaman yang telah diunduh untuk memudahkan pencarian di masa mendatang.
    \item \textbf{Ranking}: Menentukan relevansi halaman terhadap query pencarian dan mengurutkannya berdasarkan relevansi.
\end{itemize}

\subsection{Strategi Penting untuk SEO}
\begin{itemize}
    \item \textbf{Crawling}: Pastikan semua halaman web Anda dapat di-crawl oleh mesin pencari. Gunakan sitemap untuk memudahkan mesin pencari menemukan semua halaman.
    \item \textbf{Indexing}: Gunakan keywords yang relevan di halaman web Anda untuk memastikan halaman Anda terindeks dengan baik. Pilih keywords berdasarkan volume pencarian dan tingkat persaingan.
    \item \textbf{Ranking}: Peringkat di mesin pencari dipengaruhi oleh penggunaan keywords dan jumlah inlinks. Dapatkan banyak inlinks dari situs berkualitas tinggi untuk meningkatkan otoritas halaman Anda.
\end{itemize}

\subsection{Peran Blog dalam SEO}
\begin{itemize}
    \item Blog dapat meningkatkan jumlah konten di situs Anda, yang membantu dalam proses indexing dan menarik lebih banyak traffic.
\end{itemize}

\subsection{Tujuan Kampanye Iklan Online}
\begin{itemize}
    \item Tujuan kampanye iklan online dapat bervariasi, seperti akuisisi pelanggan jangka panjang, mendukung peluncuran produk, atau mencocokkan biaya per akuisisi dengan saluran iklan lainnya.
    \item Penting untuk memahami bahwa meskipun digital advertising lebih terarah dibandingkan dengan iklan offline, biaya dapat tinggi dan tidak selalu menguntungkan.
\end{itemize}

\subsection{Opsi dalam Digital Advertising}
\begin{itemize}
    \item \textbf{Search Engine Marketing}: Memposting iklan di mesin pencari seperti Google, dengan tingkat niat beli yang tinggi.
    \item \textbf{Display Ads}: Iklan gambar atau video di situs web premium atau blog, dengan jangkauan yang luas tetapi variabilitas dalam kualitas dan kinerja.
    \item \textbf{Social Ads}: Iklan di platform media sosial seperti Facebook dan Twitter, dengan banyak opsi penargetan tetapi mungkin sulit untuk mengukur manfaatnya.
\end{itemize}

\subsection{Pro dan Kontra dari Setiap Opsi Digital Ad}
\begin{itemize}
    \item Search Engine Advertising:
          \begin{itemize}
              \item Pro: Tingkat respons yang tinggi dan format yang sederhana.
              \item Kontra: Biaya yang tinggi dan informasi demografis yang terbatas.
          \end{itemize}
    \item Display Ads:
          \begin{itemize}
              \item Pro: Jangkauan yang luas dan berbagai format.
              \item Kontra: Kualitas dan kinerja yang bervariasi, serta rendahnya minat pelanggan.
          \end{itemize}
    \item Social Ads:
          \begin{itemize}
              \item Pro: Opsi penargetan yang kaya dan biaya yang lebih rendah.
              \item Kontra: Sulit untuk mengukur manfaat, terutama dalam konteks e-commerce.
          \end{itemize}
\end{itemize}

\subsection{Langkah-langkah dalam Search Engine Marketing}
\begin{enumerate}
    \item \textbf{Keyword Selection}: Mengidentifikasi kata kunci yang relevan untuk bisnis.
    \item \textbf{Ad Design}: Mendesain iklan yang jelas dan menarik.
    \item \textbf{Bidding}: Menentukan tawaran untuk setiap klik pada iklan.
\end{enumerate}

\subsection{Metrik untuk Mengukur Kinerja}
\begin{itemize}
    \item \textbf{Quantity of Exposure}: Jumlah tayangan iklan per dolar yang dibelanjakan.
    \item \textbf{Quality of Exposure}: Tingkat konversi dari klik menjadi hasil yang diinginkan.
    \item \textbf{Cost of Exposure}: Mengukur biaya per klik untuk mengevaluasi efektivitas iklan.
\end{itemize}

\subsection{Kesimpulan terkait Digital Advertising}
\begin{itemize}
    \item Digital advertising adalah aktivitas yang didorong oleh data, memerlukan eksperimen yang konstan dan fokus pada pengukuran untuk mengoptimalkan kampanye iklan. Keberhasilan dalam digital advertising bergantung pada kemampuan untuk memanfaatkan pengukuran dan penargetan secara efektif.
\end{itemize}

\subsection{Pengantar PR}
\begin{itemize}
    \item PR (Public Relations) adalah coverage editorial yang tidak dibayar di media, berbeda dengan iklan yang memerlukan biaya untuk menjangkau audiens.
    \item PR memiliki kredibilitas tinggi dan dapat menjangkau audiens yang besar secara instan.
\end{itemize}

\subsection{Keuntungan PR}
\begin{itemize}
    \item \textbf{Kesadaran Pelanggan}: PR membantu meningkatkan kesadaran pelanggan tentang produk.
    \item \textbf{Mendukung Media Lain}: Penyebutan di media dapat memicu perhatian dari media sosial dan outlet berita lainnya.
    \item \textbf{SEO}: Publikasi online meningkatkan otoritas domain dan reputasi merek di mesin pencari.
\end{itemize}

\subsection{Tantangan PR}
\begin{itemize}
    \item \textbf{Sulit untuk Diskalakan}: PR tidak dapat dengan mudah diatur atau diprediksi.
    \item \textbf{Kontrol Terbatas}: Tidak ada kontrol langsung atas bagaimana konten ditampilkan di media.
\end{itemize}

\subsection{Strategi PR}
\begin{itemize}
    \item Fokus pada media outlet yang memiliki kredibilitas tinggi, seperti New York Times.
    \item Cerita yang menarik dan relevan lebih mungkin mendapatkan perhatian dari jurnalis.
    \item Menggunakan PR firm bisa mahal, sehingga banyak startup harus melakukan PR sendiri.
\end{itemize}

\subsection{Praktik Terbaik terkait PR}
\begin{itemize}
    \item Kirim email yang personal dan singkat kepada jurnalis, menekankan apa yang menarik dari cerita Anda.
    \item Pertimbangkan untuk melakukan stunt PR yang dapat menarik perhatian, meskipun hasilnya tidak selalu dapat diprediksi.
\end{itemize}
\newpage
\section{Cost Structures, Pricing, and Tracking}
\subsection{Dasar Target Costing}
\begin{itemize}
    \item Untuk mencapai keberlanjutan finansial, jumlah yang dijual (Quantity) dikalikan dengan selisih antara harga (Price) dan biaya (Cost) harus lebih besar dari biaya tetap (Fixed Costs).
    \item Model bisnis yang berkelanjutan harus menghitung harga berdasarkan pasar dan biaya yang harus dicapai untuk memastikan profitabilitas.
\end{itemize}

\subsection{Contoh Produk: Belle-V Bottle Opener}
\begin{itemize}
    \item Belle-V bottle opener dijual dengan harga ritel $50, tetapi retailer membeli dengan harga $25, yang memberikan margin 50\% untuk retailer.
    \item Target gross margin untuk produsen ditetapkan pada 40\%, sehingga biaya maksimum yang dapat diterima untuk memproduksi produk adalah \$15.
\end{itemize}

\subsection{Perhitungan Gross Margin dan Markup}
\begin{itemize}
    \item Gross Margin dihitung sebagai (Price - Cost) / Price. Misalnya, jika harga adalah $50 dan biaya adalah $28, gross margin adalah 44\%.
    \item Markup berbeda dari gross margin dan dihitung sebagai Price / Cost - 1. Misalnya, jika harga adalah $50 dan biaya adalah $28, markup adalah 79\%.
\end{itemize}

\subsection{Faktor yang Mempengaruhi Margin Retailer}
\begin{itemize}
    \item \textbf{Volume penjualan}: Semakin tinggi volume, semakin rendah margin yang diperlukan.
    \item \textbf{Titik harga}: Produk dengan harga tinggi biasanya memiliki margin yang lebih rendah.
    \item \textbf{Diferensiasi produk}: Produk yang unik dapat memiliki margin yang lebih tinggi.
\end{itemize}

\subsection{Komponen Biaya Barang (Cost of Goods)}
\begin{itemize}
    \item \textbf{Biaya Pabrik}: Misalkan biaya pabrik untuk produk adalah \$12.
    \item \textbf{Bea Masuk}: Bea masuk untuk produk kitchen utensils adalah 3.4\% dari nilai barang saat meninggalkan pabrik. Jadi, bea masuk dihitung sebagai:
          \begin{itemize}
              \item Bea Masuk = 3.4\% x $12 = $0.408.
          \end{itemize}
    \item \textbf{Biaya Pengiriman}: Misalkan biaya pengiriman dari pabrik ke gudang di AS adalah \$0.90 per unit.
    \item Total Biaya yang Dikenakan (Landed Cost): Ini adalah total biaya yang mencakup biaya pabrik, bea masuk, dan biaya pengiriman:
          \begin{itemize}
              \item Landed Cost = Biaya Pabrik + Bea Masuk + Biaya Pengiriman
              \item Landed Cost = $12 + $0.408 + $0.90 = $13.308.
          \end{itemize}
    \item \textbf{Faktor Scrap dan Shrinkage}: Misalkan scrap dan shrinkage diperkirakan sekitar 1\% dari landed cost. Jadi, kita perlu menghitung tambahan biaya ini:
          \begin{itemize}
              \item Scrap dan Shrinkage = 1\% x $13.308 = $0.13308.
          \end{itemize}
    \item Total Cost of Goods: Akhirnya, total biaya barang (COG) dihitung dengan menambahkan landed cost dan faktor scrap/shrinkage:
          \begin{itemize}
              \item Total COG = Landed Cost + Scrap dan Shrinkage
              \item Total COG = $13.308 + $0.13308 = \$13.44108.
          \end{itemize}
\end{itemize}
Dengan perhitungan ini, total Cost of Goods untuk produk adalah sekitar \$13.44, yang masih di bawah target biaya maksimum \$15.

\subsection{Scrap dan Shrinkage}
\begin{itemize}
    \item \textbf{Scrap} merujuk pada produk yang tidak dapat dijual karena cacat atau kerusakan selama proses produksi. Misalnya, jika Anda memproduksi botol pembuka dan beberapa dari mereka memiliki goresan atau kerusakan, botol-botol tersebut dianggap sebagai scrap. Scrap biasanya diukur sebagai persentase dari total produk yang diproduksi.

          Menentukan Scrap:
          \begin{itemize}
              \item Untuk menentukan tingkat scrap, Anda perlu menghitung jumlah produk yang cacat dibandingkan dengan total produk yang diproduksi. Misalnya, jika Anda memproduksi 1000 botol pembuka dan 50 di antaranya cacat, maka tingkat scrap adalah:
                    \begin{align*}
                        Tingkat Scrap & = \frac{Jumlah Scrap}{Total Produksi} \times 100 \\
                        Tingkat Scrap & = \frac{50}{1000} \times 100 = 5\%
                    \end{align*}

          \end{itemize}
    \item \textbf{Shrinkage} adalah istilah yang digunakan untuk menggambarkan kehilangan produk yang terjadi karena pencurian, kerusakan, atau kesalahan dalam pengelolaan inventaris. Ini bisa terjadi di gudang atau selama proses distribusi. Shrinkage juga diukur sebagai persentase dari total inventaris.

          Menentukan Shrinkage:
          \begin{itemize}
              \item Untuk menentukan tingkat shrinkage, Anda perlu menghitung jumlah produk yang hilang dibandingkan dengan total inventaris yang seharusnya ada. Misalnya, jika Anda memiliki 1000 botol pembuka dalam inventaris tetapi hanya menemukan 950, maka tingkat shrinkage adalah:
                    \begin{align*}
                        \text{Tingkat Shrinkage} & = \frac{Jumlah Hilang}{Total Inventaris} \times 100 \\
                        \text{Tingkat Shrinkage} & = \frac{50}{1000} \times 100 = 5\%
                    \end{align*}
          \end{itemize}
\end{itemize}

\subsection{Pentingnya Penetapan Harga}
\begin{itemize}
    \item Penetapan harga adalah salah satu keputusan terpenting yang diambil oleh seorang entrepreneur.
    \item Fokus tidak hanya pada harga, tetapi juga pada model pendapatan yang tepat.
\end{itemize}

\subsection{Empat Input Kunci dalam Keputusan Harga}
\begin{itemize}
    \item \textbf{Cost}: Biaya produksi memberikan batas bawah untuk harga. Ini mencakup semua biaya yang dikeluarkan untuk membuat produk.
    \item \textbf{Willingness to Pay (WTP)}: Ini adalah batas atas harga, yaitu seberapa banyak pelanggan bersedia membayar untuk produk. Ini juga dikenal sebagai harga reservasi.
    \item \textbf{Competition}: Persaingan mempengaruhi harga yang dapat ditetapkan. Kita mungkin perlu menetapkan harga lebih rendah dari WTP maksimum untuk menarik pelanggan.
    \item \textbf{Channel Partners}: Mitra saluran, seperti distributor atau pengecer, dapat mempengaruhi keputusan harga. Mereka perlu mendapatkan keuntungan, sehingga harga yang ditetapkan sering kali lebih tinggi untuk memastikan semua pihak mendapatkan imbalan.
\end{itemize}

\subsection{Apa itu Elastisitas?}
\begin{itemize}
    \item Elastisitas adalah ukuran seberapa responsif suatu variabel terhadap perubahan dalam variabel lain. Dalam konteks harga, elastisitas menunjukkan seberapa besar perubahan dalam permintaan (unit sales) atau revenue ketika harga berubah. Ada tiga jenis elastisitas yang sering dibahas:
          \begin{itemize}
              \item Price Elasticity: Mengukur seberapa banyak unit sales berubah ketika harga berubah. Ini membantu kita memahami apakah permintaan untuk suatu produk sensitif terhadap perubahan harga.
              \item Revenue Elasticity: Mengukur seberapa banyak revenue berubah ketika harga berubah. Ini menunjukkan dampak perubahan harga terhadap total pendapatan yang dihasilkan dari penjualan.
              \item Break-even Profit Elasticity: Ini adalah elastisitas yang diperlukan untuk mencapai titik impas dalam profit. Ini menunjukkan seberapa besar perubahan dalam unit sales yang diperlukan untuk menutupi biaya setelah perubahan harga.
          \end{itemize}
\end{itemize}

\subsection{Definisi Price Elasticiy}
\begin{itemize}
    \item \textbf{Price elasticity} adalah ukuran seberapa besar perubahan persentase dalam unit sales ketika terjadi perubahan persentase dalam harga.
    \item Definisi formalnya adalah
          \begin{align*}
              \frac{\% \text{perubahan dalam unit sales}}{\% \text{perubahan dalam harga}}.
          \end{align*}
\end{itemize}

\subsection{Mengapa Menggunakan Persentase?}
\begin{itemize}
    \item Menggunakan persentase memungkinkan kita untuk menghilangkan pengaruh unit pengukuran, sehingga price elasticity menjadi angka murni yang tidak tergantung pada satuan seperti dolar atau ton.
\end{itemize}

\subsection{Elastic dan Inelastic}
\begin{itemize}
    \item \textbf{Elastic demand} terjadi ketika penurunan harga menyebabkan peningkatan revenue, sedangkan \textbf{inelastic demand} terjadi ketika penurunan harga menyebabkan penurunan revenue. Dan juga sebaliknya untuk masing-masing istilah.
    \item Revenue elasticity didefinisikan sebagai persentase perubahan dalam revenue dibagi dengan persentase perubahan dalam harga.
\end{itemize}

\subsection{Break-even Profit Elasticity}
\begin{itemize}
    \item \textbf{Break-even profit elasticity} adalah ukuran seberapa banyak unit sales harus meningkat untuk menutupi penurunan profit akibat perubahan harga, yang berbeda untuk setiap jenis bisnis.
    \item Untuk revenue, break-even elasticity selalu -1, tetapi untuk profit, elastisitas ini tergantung pada struktur biaya bisnis.
    \item Memahami break-even profit elasticity membantu perusahaan membuat keputusan yang lebih baik tentang penetapan harga. Jika data menunjukkan bahwa permintaan tidak akan meningkat cukup untuk menutupi penurunan harga, maka perusahaan mungkin harus mempertimbangkan untuk tidak menurunkan harga.
\end{itemize}

\subsection{Contoh Kasus Analisa Elastisitas}
\begin{itemize}
    \item Dalam contoh Admiral Electric, jika mereka mempertimbangkan penurunan harga sebesar 5\%, analisis elastisitas diperlukan untuk menentukan dampaknya terhadap profit.
    \item Elastisitas yang diperlukan untuk mencapai break-even profit bervariasi tergantung pada margin kontribusi produk.
\end{itemize}

\subsection{Kesimpulan terkait Elastisitas}
\begin{itemize}
    \item Tiga konsep utama yang dibahas adalah \textbf{price elasticity, revenue elasticity, dan break-even profit elasticity}. Memahami konsep-konsep ini sangat penting untuk membuat keputusan harga yang baik.
\end{itemize}

\subsection{Definisi dan Konteks Price Elasticity}
\begin{itemize}
    \item \textbf{Price elasticity} mengacu pada seberapa besar respons konsumen terhadap perubahan harga, baik itu perubahan harga permanen atau sementara.
    \item Respons konsumen biasanya lebih tinggi terhadap perubahan harga jangka pendek dibandingkan dengan perubahan harga jangka panjang.
\end{itemize}

\subsection{Jenis Price Elasticity}
\begin{itemize}
    \item \textbf{Price Elasticity of Demand (PED)}: Mengukur seberapa besar perubahan jumlah yang diminta terhadap perubahan harga. Jika PED > 1, permintaan elastis; jika PED < 1, permintaan inelastis.
    \item \textbf{Income Elasticity of Demand}: Mengukur seberapa besar perubahan jumlah yang diminta terhadap perubahan pendapatan konsumen.
    \item \textbf{Cross-Price Elasticity of Demand}: Mengukur seberapa besar perubahan jumlah yang diminta untuk satu produk sebagai respons terhadap perubahan harga produk lain.
\end{itemize}

\subsection{Faktor yang Mempengaruhi Price Elasticity:}
\begin{itemize}
    \item Ketersediaan Substitusi: Semakin banyak produk pengganti yang tersedia, semakin elastis permintaan.
    \item Proporsi Pendapatan: Produk yang memakan proporsi besar dari pendapatan konsumen cenderung memiliki permintaan yang lebih elastis.
    \item Kebutuhan vs. Keinginan: Barang kebutuhan cenderung memiliki permintaan yang inelastis, sedangkan barang keinginan lebih elastis.
\end{itemize}

\subsection{Faktor yang Mempengaruhi Respons Konsumen}
\begin{itemize}
    \item Penawaran diskon harga jangka pendek sering kali mendorong konsumen untuk beralih merek, membeli lebih awal, atau bahkan menimbun produk.
    \item Namun, dalam beberapa kasus, seperti harga gasoline, permintaan mungkin tidak elastis terhadap perubahan harga jangka pendek karena konsumen sudah memiliki kendaraan.
\end{itemize}

\subsection{Metode Pengukuran Price Elasticity}
\begin{itemize}
    \item Ada berbagai cara untuk mengukur price sensitivity, termasuk survei, eksperimen terkontrol, dan analisis data historis.
    \item Pengukuran dapat dilakukan di tingkat pasar atau individu, dengan respons pasar biasanya lebih halus dibandingkan dengan respons individu.
\end{itemize}

\subsection{Pendekatan Pengukuran}
\begin{itemize}
    \item Metode pengukuran dapat dibagi menjadi empat kategori berdasarkan apa yang diukur (actual purchase vs. intentions) dan kondisi pengukuran (natural vs. controlled settings).
    \item Contoh metode: Menggunakan survei untuk mengumpulkan data dari kelompok konsumen yang berbeda mengenai respons mereka terhadap harga produk baru.
    \item Memahami konteks di mana price elasticity diukur sangat penting, termasuk faktor-faktor seperti waktu, lokasi, dan kondisi pasar.
\end{itemize}

\subsection{Keterbatasan Metode}
\begin{itemize}
    \item Metode survei tidak selalu dapat menangkap perilaku pembelian aktual dan hanya mengukur niat untuk membeli.
    \item Penting untuk memahami hubungan antara niat dan perilaku pembelian untuk meningkatkan akurasi prediksi.
\end{itemize}

\subsection{Pengantar Data Historis dan Price Elasticity}
\begin{itemize}
    \item Data diambil dari studi kasus HBS tentang FedEx yang meluncurkan layanan pengiriman dokumen baru, Courier Pack, dengan data harga dan penjualan unit selama 29 minggu.
    \item Perusahaan menggunakan strategi penetration pricing, dimulai dengan harga rendah dan meningkat seiring waktu.
\end{itemize}

\subsection{Analisis Regresi dan Koefisien Harga}
\begin{itemize}
    \item  Analisis regresi dilakukan untuk menghubungkan penjualan unit dengan harga, menghasilkan koefisien harga positif yang menunjukkan bahwa peningkatan harga dapat meningkatkan penjualan unit, meskipun ini mungkin tidak akurat.
    \item Penting untuk mempertimbangkan faktor lain seperti peningkatan iklan dan efek word of mouth yang mungkin mempengaruhi penjualan.
\end{itemize}

\subsection{Menghitung Elasticity}
\begin{itemize}
    \item Dengan memasukkan waktu sebagai covariant dalam analisis regresi, koefisien harga menjadi negatif, menunjukkan bahwa peningkatan harga dapat mengurangi penjualan unit.
    \item Elasticity dihitung dengan membandingkan persentase perubahan penjualan unit dengan persentase perubahan harga, menghasilkan nilai elasticity yang menunjukkan bahwa permintaan bersifat inelastic.
\end{itemize}

\subsection{Implikasi untuk Keputusan Bisnis}
\begin{itemize}
    \item Jika permintaan inelastic, peningkatan harga dapat meningkatkan pendapatan.
    \item Penting untuk mempertimbangkan apakah elasticity yang dihitung dari data historis akan berlaku di masa depan dan apakah ada perubahan signifikan dalam elasticity selama periode analisis.
\end{itemize}

\subsection{Metode Pengukuran Price Elasticity}
\begin{itemize}
    \item Terdapat berbagai metode untuk mengukur price elasticity, termasuk analisis data historis dan eksperimen terkontrol.
    \item Data historis dapat memberikan estimasi yang baik jika dianalisis dengan hati-hati, tetapi harus ada kontrol yang baik saat melampaui rentang operasi yang ada.
\end{itemize}

\subsection{Penndahuluan Maximum Willingness to Pay (MWTP)}
\begin{itemize}
    \item Memperkirakan \textbf{maximum willingness to pay} adalah kunci dalam pengambilan keputusan harga.
    \item Variasi dalam willingness to pay tercermin dalam \textbf{demand function}.
\end{itemize}

\subsection{Pendekatan untuk Mengestimasi Maximum Willingness to Pay}
\begin{itemize}
    \item \textbf{Economic Value to the Customer (EVC)}: Metode ini berguna dalam konteks bisnis-ke-bisnis ketika survei tidak memungkinkan.
    \item Contoh: Menghitung berapa banyak Professor Raju bersedia membayar untuk tablet baru berdasarkan biaya metode lama.
\end{itemize}

\subsection{Menghitung Economic Value to the Customer}
\begin{itemize}
    \item EVC dihitung dengan mempertimbangkan biaya yang dikeluarkan untuk metode lama dan membandingkannya dengan harga produk baru.
    \item EVC memberikan angka yang membantu dalam menentukan harga dan mengidentifikasi segmen pasar yang berbeda.
\end{itemize}

\subsection{Variabel yang Mempengaruhi Willingness to Pay}
\begin{itemize}
    \item Variabel yang dapat mempengaruhi willingness to pay termasuk ukuran kolam renang, kesadaran lingkungan, dan lokasi geografis.
    \item Penting untuk menemukan variabel yang dapat diamati dan dapat ditindaklanjuti untuk menentukan siapa yang bersedia membayar lebih.
\end{itemize}

\subsection{Customer Lifetime Value (CLV)}
\begin{itemize}
    \item \textbf{Customer Lifetime Value (CLV)} adalah nilai total yang diharapkan dari seorang pelanggan selama masa hubungan mereka dengan perusahaan. CLV membantu bisnis memahami seberapa banyak pendapatan yang dapat dihasilkan dari seorang pelanggan selama mereka bertransaksi dengan perusahaan.
    \item Mengapa CLV Penting: Memahami CLV membantu perusahaan dalam pengambilan keputusan strategis, seperti berapa banyak yang harus diinvestasikan dalam pemasaran dan akuisisi pelanggan. Dengan mengetahui nilai pelanggan, perusahaan dapat lebih baik menargetkan segmen pasar yang paling menguntungkan.
\end{itemize}

\subsection{Kesimpulan terkait WTP}
\begin{itemize}
    \item EVC tidak hanya memberikan angka, tetapi juga wawasan tentang variasi willingness to pay di antara pelanggan.
    \item Menggabungkan EVC dengan customer lifetime value membantu memahami basis pelanggan dan strategi pemasaran yang lebih baik.
\end{itemize}

\subsection{Metode Van Westendorp}
\begin{itemize}
    \item Definisi: Metode yang digunakan untuk menentukan harga yang dianggap wajar oleh konsumen berdasarkan pertanyaan yang diajukan.
    \item Empirical Basis: Metode ini didasarkan pada data empiris dan sering digunakan dalam praktik oleh perusahaan riset pasar.
\end{itemize}

\subsection{Pertanyaan Kunci pada Metode Van W.}
\begin{itemize}
    \item Empat pertanyaan yang diajukan kepada responden untuk mengumpulkan data tentang harga.
    \item \begin{itemize}
              \item Bargain Price: Pada harga berapa Anda menganggap Produk X adalah tawaran yang baik?
              \item Getting Expensive: Pada harga berapa Anda menganggap Produk X mulai mahal tetapi masih dipertimbangkan untuk dibeli?
              \item Too Expensive: Pada harga berapa Anda menganggap Produk X terlalu mahal untuk dipertimbangkan?
              \item Too Cheap: Pada harga berapa Anda menganggap Produk X terlalu murah sehingga Anda meragukan apakah itu akan berfungsi?
          \end{itemize}
\end{itemize}

\subsection{Analisis Data dalam Metode Van W.}
\begin{itemize}
    \item Definisi: Mengumpulkan dan menganalisis jawaban dari responden untuk menentukan rentang harga.
    \item Plotting: Data yang dikumpulkan kemudian dipetakan untuk menemukan titik-titik penting seperti:
          \begin{itemize}
              \item \textbf{Marginal Cheapness}: Titik di mana harga terlalu murah bertemu dengan harga yang dianggap mahal.
              \item \textbf{Marginal Expensiveness}: Titik di mana harga yang dianggap mahal bertemu dengan harga yang dianggap tawaran yang baik.
              \item \textbf{Indifference Point}: Titik di mana jumlah responden yang menganggap harga mahal sama dengan yang menganggapnya tawaran yang baik.
              \item \textbf{Optimal Price Point}: Titik di mana jumlah responden yang menganggap harga terlalu murah sama dengan yang menganggapnya terlalu mahal.
          \end{itemize}
\end{itemize}

\subsection{Pengantar Conjoint Analysis}
\begin{itemize}
    \item Perusahaan dapat menentukan harga produk dengan menggunakan analisis conjoint untuk memahami fitur yang diinginkan pelanggan dan seberapa banyak mereka bersedia membayar.
    \item Conjoint analysis adalah alat yang kuat untuk mengidentifikasi segmen pasar dan preferensi pelanggan.
    \item Definisi Istilah Teknis
          \begin{itemize}
              \item Maximum Willingness to Pay (MWTP): Jumlah maksimum yang bersedia dibayar oleh pelanggan untuk suatu produk.
              \item Conjoint Analysis: Metode untuk memahami preferensi pelanggan dengan menganalisis kombinasi atribut produk.
              \item Product Profile: Deskripsi lengkap dari produk yang mencakup semua atribut yang relevan.
              \item Regression Analysis: Teknik statistik untuk memahami hubungan antara variabel independen dan dependen.
              \item Utility: Kepuasan atau nilai yang diperoleh pelanggan dari produk.
          \end{itemize}
\end{itemize}

\subsection{Langkah-langkah dalam Conjoint Analysis}
\begin{itemize}
    \item Identifikasi Atribut: Langkah pertama adalah mengidentifikasi atribut penting bagi pelanggan. Dalam contoh ini, atribut yang dipertimbangkan adalah:
          \begin{itemize}
              \item Brand Name: Nama merek produk.
              \item Storage Capacity: Kapasitas penyimpanan MP3 player.
              \item Battery Life: Lama waktu baterai bertahan.
              \item Display: Apakah tampilan berwarna atau monokrom.
              \item Warranty: Apakah ada garansi atau tidak.
              \item Price: Harga produk.
          \end{itemize}
    \item Tentukan Rentang dan Level Atribut: Setiap atribut harus memiliki rentang dan level yang disepakati. Misalnya, kapasitas penyimpanan berkisar dari 50 hingga 5,000 lagu.
\end{itemize}

\subsection{Pengumpulan Data dalam Conjoint Analysis}
\begin{itemize}
    \item Product Profiles: Buat profil produk yang menggambarkan kombinasi atribut. Misalnya, profil produk bisa berupa:
          \begin{itemize}
              \item Brand: Orange, Storage: 5,000 songs, Battery: 18 hours, Display: Color, Warranty: None, Price: \$249.
          \end{itemize}
    \item Rating Profiles: Responden diminta untuk memberikan rating pada berbagai profil produk. Metode pengumpulan data bisa berupa ranking, rating, atau perbandingan pasangan.
\end{itemize}

\subsection{Analisis Data dalam Conjoint Analysis}
\begin{itemize}
    \item \textbf{Regresi}: Setelah data dikumpulkan, analisis regresi digunakan untuk mengestimasi utilitas berdasarkan rating yang diberikan oleh responden. Koefisien dari regresi menunjukkan seberapa besar pengaruh setiap atribut terhadap utilitas.
\end{itemize}

\subsection{Menghitung Maximum WTP}
\begin{itemize}
    \item \textbf{Konversi Koefisien ke Dollar}: Koefisien dari analisis regresi diterjemahkan ke dalam nilai dollar untuk menghitung MWTP. Misalnya, jika 33.6 unit utilitas setara dengan \$150, maka satu unit utilitas bernilai \$4.45.
    \item \textbf{Estimasi MWTP untuk Model}: Dengan menggunakan koefisien, total utilitas untuk setiap model produk dihitung, yang kemudian dikonversi menjadi MWTP.
\end{itemize}

\subsection{Pengantar tentang KPIs (Key Performance Indicator)}
\begin{itemize}
    \item KPIs adalah metrik yang digunakan untuk mengukur kinerja suatu bisnis dan membantu dalam pengambilan keputusan.
    \item Penting untuk melaporkan KPIs kepada investor dan untuk mengelola pertumbuhan bisnis.
\end{itemize}

\subsection{Contoh Kasus KPI: Bandar Foods}
\begin{itemize}
    \item Operating Metric 1 (OM1): Jumlah SKUs (Stock Keeping Units) - Ini adalah jumlah produk yang dijual.
          \begin{itemize}
              \item Definisi: SKU adalah identifikasi unik untuk setiap produk yang dijual.
          \end{itemize}
    \item Operating Metric 2 (OM2): Jumlah retailer - Ini mengukur berapa banyak toko yang menjual produk mereka.
          \begin{itemize}
              \item Definisi: Retailer adalah toko atau perusahaan yang menjual produk langsung kepada konsumen.
          \end{itemize}
    \item Operating Metric 3 (OM3): Sales velocity - Ini mengukur jumlah unit yang terjual per SKU per toko per minggu.
          \begin{itemize}
              \item Definisi: Sales velocity adalah kecepatan penjualan produk di toko.
          \end{itemize}
\end{itemize}

\subsection{Contoh Kasus KPI: Gridium}
\begin{itemize}
    \item Gridium menggunakan dashboard yang disebut "State of Biz" untuk melaporkan KPIs kepada investor.
    \item Metrik yang dilaporkan termasuk:
          \begin{itemize}

              \item Monthly Recurring Revenue: Pendapatan yang dihasilkan secara berulang setiap bulan.
                    \begin{itemize}
                        \item Definisi: Pendapatan yang dihasilkan dari langganan yang dibayar secara berkala.
                    \end{itemize}
              \item Revenue by Geography: Pendapatan yang dipecah berdasarkan lokasi.
              \item Monthly Recurring Revenue per Account: Pendapatan per akun pelanggan.
              \item Customer and Building Counts: Jumlah pelanggan dan bangunan yang menggunakan produk.
          \end{itemize}
\end{itemize}

\subsection{Metode Pelaporan Lain}
\begin{itemize}
    \item Gridium juga menggunakan metode pelaporan kualitatif, seperti indikator lampu lalu lintas (traffic light) untuk menunjukkan kemajuan terhadap tujuan.
          \begin{itemize}
              \item Definisi: Metode ini menggunakan warna (hijau, kuning, merah) untuk menunjukkan status pencapaian tujuan.
          \end{itemize}
\end{itemize}

\subsection{Kesimpulan terkait KPI}
\begin{itemize}
    \item Penting untuk memilih 3 hingga 5 metrik kunci yang benar-benar mencerminkan kesehatan bisnis.
    \item KPIs harus dilaporkan secara rutin untuk membantu pengelolaan bisnis dan memberikan transparansi kepada investor.
\end{itemize}

\subsection{Sales Partnerships}
\begin{itemize}
    \item Definisi: Kerjasama antara dua perusahaan di mana satu perusahaan membantu menjual produk perusahaan lain kepada pelanggan mereka.
    \item Tujuan: Mencapai pelanggan yang sudah ada dalam jaringan mitra dan menambah nilai bagi mitra tersebut.
    \item Istilah Teknis:
          \begin{itemize}
              \item Sales Partnerships: Kerjasama antara dua perusahaan untuk menjual produk satu sama lain.
              \item Customer Acquisition: Proses mendapatkan pelanggan baru untuk produk atau layanan.
              \item Revenue Sharing: Pembagian pendapatan antara dua pihak yang terlibat dalam kerjasama.
          \end{itemize}
\end{itemize}

\subsection{Contoh Kerjasama Sales Partnerships}
\begin{enumerate}
    \item \textbf{Ford Motor Company}
          \begin{itemize}
              \item Strategi: Menawarkan TerraPass saat pembelian mobil untuk mengurangi dampak lingkungan dari kepemilikan mobil.
              \item Manfaat: Ford mendapatkan pendapatan tambahan dan akses ke informasi pelanggan.
          \end{itemize}
    \item \textbf{Expedia}
          \begin{itemize}
              \item Strategi: Menawarkan TerraPass sebagai tambahan saat checkout setelah pembelian tiket pesawat.
              \item Manfaat: TerraPass mendapatkan hampir satu juta pelanggan baru, dan Expedia meningkatkan citra lingkungan mereka.
          \end{itemize}
    \item \textbf{Bloomsberry Chocolate}
          \begin{itemize}
              \item Strategi: Menggabungkan sertifikat TerraPass dengan pembelian cokelat untuk mengurangi jejak lingkungan selama satu hari.
              \item Menciptakan nilai tambah bagi produk cokelat dan menjangkau pelanggan baru.
          \end{itemize}
\end{enumerate}

\subsection{Keuntungan Sales Partnerships}
\begin{itemize}
    \item \textbf{Akses ke Segmen Pelanggan}: Mitra sudah memiliki akses ke pelanggan yang relevan.
    \item \textbf{Nilai Tambah}: Menawarkan atribut merek yang diinginkan oleh mitra, seperti keberlanjutan lingkungan.
\end{itemize}

\subsection{Keputusan Make or Buy}
\begin{itemize}
    \item Definisi: Keputusan apakah perusahaan harus memproduksi barang sendiri (make) atau membelinya dari pihak ketiga (buy).
    \item Harry's memilih untuk membeli pabrik pembuatan pisau cukur daripada membeli dari pasar terbuka, yang mungkin mengejutkan bagi banyak orang.
\end{itemize}

\subsection{Alasan untuk Membeli dari Pihak Ketiga}
\begin{itemize}
    \item \textbf{Avoid Duplicate Investment}: Menghindari investasi yang tidak perlu jika ada pasokan yang kompetitif di pasar.
    \item \textbf{Division of Labor}: Perusahaan sering kali lebih baik dalam satu aktivitas dalam rantai nilai, seperti inovasi, dan tidak efisien dalam aktivitas lain, seperti distribusi.
    \item \textbf{Pay for Goods, Not Effort}: Dengan membeli dari pemasok, perusahaan hanya membayar untuk barang yang memenuhi spesifikasi kualitas, bukan untuk usaha yang dilakukan.
\end{itemize}

\subsection{Faktor yang Meningkatkan Biaya Transaksi}
\begin{itemize}
    \item \textbf{Specificity of Investment}: Jika investasi sangat spesifik, perusahaan bisa terjebak dalam negosiasi ulang kontrak.
    \item \textbf{Complexity of Transaction}: Transaksi yang kompleks, seperti kontrak penelitian, sulit untuk diukur dan dinilai.
    \item \textbf{Difficulty of Measuring Performance}: Mengukur kinerja dalam transaksi yang kompleks bisa menjadi tantangan.
\end{itemize}

\subsection{Faktor yang Menurunkan Biaya Transaksi}
\begin{itemize}
    \item \textbf{Frequency and Duration of Transactions}: Hubungan yang sering dan lama dapat menurunkan biaya transaksi karena reputasi dan kepercayaan.
    \item \textbf{Connectedness of Social Structure}: Keterhubungan sosial dapat mempengaruhi perilaku dan keputusan dalam transaksi.
\end{itemize}

\subsection{Manfaat dari Keputusan Make
}
\begin{itemize}
    \item \textbf{Avoid Transaction Costs}: Menghindari semua kerumitan yang terkait dengan kontrak dengan pihak ketiga.
    \item \textbf{Control and Learning Curves}: Mengendalikan proses produksi dan belajar dari pengalaman untuk meningkatkan efisiensi.
    \item \textbf{Investasi Jangka Panjang}: Meskipun membeli pabrik memerlukan investasi awal yang besar, mungkin dapat melihatnya sebagai langkah strategis jangka panjang untuk membangun merek dan meningkatkan daya saing di pasar.
\end{itemize}

\newpage
\section{Creating and Scaling Company Culture}
\subsection{Pentingnya Hiring dalam Startup}
\begin{itemize}
    \item Hiring yang tepat sangat krusial untuk kesuksesan perusahaan, terutama bagi para entrepreneur yang mungkin belum berpengalaman dalam proses ini.
    \item Terdapat perbedaan besar antara karyawan yang berkualitas tinggi dan yang biasa-biasa saja, yang dapat mempengaruhi kinerja perusahaan.
    \item Definisi Istilah Teknis
          \begin{itemize}
              \item Hiring: Proses merekrut karyawan untuk mengisi posisi dalam perusahaan.
              \item Retention: Upaya untuk mempertahankan karyawan agar tetap bekerja di perusahaan.
              \item IRR (Internal Rate of Return): Ukuran profitabilitas dari investasi, dalam konteks ini, efektivitas teknik interview.
              \item Scorecard: Alat untuk mencatat kriteria yang dicari dalam kandidat selama proses hiring.
              \item Homophily: Kecenderungan untuk lebih menyukai orang yang memiliki kesamaan dengan diri sendiri.
          \end{itemize}
\end{itemize}

\subsection{Teknik Interview yang Efektif}
\begin{itemize}
    \item Artist: Mengandalkan penilaian intuitif tanpa data. IRR (Internal Rate of Return) untuk pendekatan ini adalah 25\%, yang rendah.
    \item Sponge: Mengumpulkan data tanpa rencana yang jelas. IRR-nya lebih rendah lagi, yaitu 20\%.
    \item Prosecutor: Menginterogasi kandidat dengan agresif. Ini adalah pendekatan terburuk dengan IRR hanya 10\%.
    \item Spy (Infiltrator): Menghabiskan waktu bersama kandidat untuk memahami mereka secara mendalam. IRR-nya sangat tinggi, yaitu 100\%, tetapi tidak praktis.
    \item Airline Pilot: Menggunakan checklist untuk menilai kandidat. Ini adalah pendekatan paling efektif dengan IRR 80\%.
\end{itemize}

\subsection{Checklist dan Scorecard}
\begin{itemize}
    \item Membuat scorecard untuk mencatat kriteria yang dicari dalam kandidat. Ini membantu dalam membandingkan kandidat secara objektif.
    \item Menghindari bias homophily, yaitu kecenderungan untuk lebih menyukai orang yang mirip dengan diri sendiri, yang dapat mengurangi keragaman dalam tim.
\end{itemize}

\subsection{Proses Interview}
\begin{itemize}
    \item Melibatkan 3-5 orang dalam proses interview untuk mendapatkan berbagai sudut pandang.
    \item Fokus pada pertanyaan berbasis sejarah untuk menggali pengalaman kandidat, bukan pertanyaan teka-teki yang tidak relevan.
\end{itemize}

\subsection{Feedback dan Performance Review}
\begin{itemize}
    \item Penting untuk memiliki proses review kinerja yang teratur, meskipun untuk tim kecil.
    \item Proses feedback harus spesifik dan berkelanjutan, serta melibatkan semua anggota tim.
\end{itemize}

\subsection{Pentingnya Budaya Perusahaan}
\begin{itemize}
    \item Budaya perusahaan yang baik sangat penting untuk kesuksesan startup. Jika tidak diperhatikan sejak awal, budaya yang terbentuk mungkin tidak sesuai dengan yang diinginkan.
    \item Definisi Istilah Teknis
          \begin{itemize}
              \item Attachment: Alasan mengapa karyawan bekerja di perusahaan.
              \item Retention: Kemampuan perusahaan untuk mempertahankan karyawan.
              \item Cultural Fit: Kesesuaian antara nilai dan budaya karyawan dengan perusahaan.
              \item Direct Monitoring: Pengawasan langsung terhadap karyawan untuk memastikan kinerja.
              \item Professional Standards: Standar yang ditetapkan oleh profesi untuk mengontrol kinerja.
          \end{itemize}
\end{itemize}

\subsection{Tiga Menu Utama dalam Budaya Perusahaan}
\begin{enumerate}
    \item Basis Attachment \& Retention
          \begin{itemize}
              \item Compensation: Alasan orang bekerja adalah untuk mendapatkan gaji.
              \item Quality of Task: Orang bekerja karena pekerjaan itu menarik dan menantang.
              \item Love: Orang bekerja karena mereka menyukai lingkungan dan rekan kerja mereka.
          \end{itemize}
    \item Kriteria Seleksi Karyawan
          \begin{itemize}
              \item Skills: Memilih karyawan berdasarkan keterampilan yang dibutuhkan saat ini.
              \item Potential: Memilih karyawan berdasarkan potensi mereka untuk berkembang di masa depan.
              \item Cultural Fit: Memilih karyawan yang sesuai dengan budaya perusahaan yang ada.
          \end{itemize}
    \item Koordinasi dan Kontrol
          \begin{itemize}
              \item Direct Monitoring: Memantau karyawan secara langsung untuk memastikan mereka melakukan pekerjaan dengan benar.
              \item Peer or Cultural Control: Mengandalkan tekanan sosial dari rekan kerja untuk mendorong kinerja.
              \item Professional Standards: Mengandalkan standar profesional untuk mengontrol kinerja.
          \end{itemize}
\end{enumerate}

\subsection{Lima Model Stabil dalam Budaya Perusahaan}
\begin{itemize}
    \item Star Model
          \begin{itemize}
              \item Attachment: Quality of Task
              \item Selection: Exceptional Potential
              \item Control: Professional
          \end{itemize}
    \item Commitment Model
          \begin{itemize}
              \item Attachment: Love
              \item Selection: Cultural Fit
              \item Control: Cultural
          \end{itemize}
    \item Engineering Model
          \begin{itemize}
              \item Attachment: Quality of Task
              \item Selection: Skills
              \item Control: Peer
          \end{itemize}
    \item Bureaucracy Model
          \begin{itemize}
              \item Attachment: Quality of Task
              \item Selection: Skills
              \item Control: Formal Procedures
          \end{itemize}
    \item Autocracy Model
          \begin{itemize}
              \item Attachment: Compensation
              \item Selection: Skills
              \item Control: Direct Monitoring
          \end{itemize}
\end{itemize}

\subsection{Kesimpulan terkait Budaya Perusahaan}
\begin{itemize}
    \item Model yang dipilih akan mempengaruhi kesuksesan perusahaan. Model Commitment dan Star adalah yang paling efektif, tetapi juga paling sulit untuk diterapkan. Penting untuk mempertimbangkan budaya perusahaan sejak awal untuk mencapai pertumbuhan yang sukses.
\end{itemize}

\subsection{Pertumbuhan Startup}
\begin{itemize}
    \item Banyak startup yang sukses mengalami kegagalan saat fase pertumbuhan, baik karena tumbuh terlalu cepat atau dengan cara yang salah.
    \item Penting untuk memahami bahwa tidak semua startup harus tumbuh; beberapa pemilik bisnis lebih memilih untuk membangun organisasi yang stabil.
\end{itemize}

\subsection{Keuntungan dari Pertumbuhan}
\begin{itemize}
    \item Perusahaan yang berusaha untuk tumbuh cenderung bertahan lebih lama karena mereka dapat mempertahankan talenta yang baik.
    \item Pertumbuhan memberikan pendapatan tambahan yang dapat digunakan untuk beradaptasi dengan perubahan situasi.
\end{itemize}

\subsection{Model Pertumbuhan Churchill dan Lewis}
\begin{itemize}
    \item
\end{itemize}

\subsection{Vertical Expansion}
\begin{itemize}
    \item Vertical expansion adalah strategi di mana perusahaan tetap berada dalam industri yang sama tetapi memperluas ke geografi atau segmen yang berbeda.
    \item Contoh: easyJet yang sukses di Eropa dapat mempertimbangkan untuk memperluas ke Asia atau Afrika dengan model bisnis yang sama.
\end{itemize}

\subsection{Horizontal Expansion}
\begin{itemize}
    \item Horizontal expansion adalah strategi di mana perusahaan memasuki industri yang berbeda dari basis industri mereka.
    \item Contoh: Walt Disney yang awalnya adalah studio film, kemudian memperluas ke taman hiburan seperti Disneyland.
\end{itemize}

\subsection{Pertimbangan dalam Ekspansi}
\begin{itemize}
    \item \textbf{Core Capabilities}: Kemampuan inti yang membedakan perusahaan dari pesaing dan sulit untuk ditiru.
    \item \textbf{Fit or Consistency}: Apakah aktivitas baru sejalan dengan apa yang sudah dilakukan perusahaan sebelumnya.
\end{itemize}

\subsection{Pengujian Ekspansi Horizontal}
\begin{itemize}
    \item \textbf{Better Off Test}: Apakah menggabungkan aktivitas di bawah satu atap organisasi akan menciptakan lebih banyak nilai dibandingkan jika tidak memiliki kepemilikan bersama.
    \item \textbf{Best Alternative Test}: Apakah ada alternatif yang lebih baik untuk mengorganisir aktivitas tersebut, seperti lisensi atau aliansi strategis.
\end{itemize}

\subsection{Kesimpulan terkait Ekspansi}
\begin{itemize}
    \item Perusahaan harus mempertimbangkan baik vertical maupun horizontal expansion dengan memperhatikan kemampuan inti mereka dan konsistensi dalam aktivitas yang dilakukan. Keputusan ini akan memandu jalur ekspansi mereka ke depan.
\end{itemize}

% END Of DOC
\end{document}