% section | subsection | subsubsection | paragraph | subparagraph | verbatim | enumerate | item

% Just let it be an Article bro
\documentclass{article}

% Cover Metadata
\title{Financing and Profitability}
\date{12 Oktober 2025}
\author{Maulana Hafidz Ismail}

% Equation Setup
\usepackage{amsmath}

% Image Setup
\usepackage{graphicx}
\usepackage{float}

% Link Setup
\usepackage{hyperref}

% Highlighting Setup
\usepackage{soul}
\usepackage{xcolor}
\sethlcolor{red!30}
\newcommand{\hlblue}[1]{\sethlcolor{cyan!30}\hl{#1}\sethlcolor{yellow}}

% Line Width
\sloppy
\setlength{\emergencystretch}{3em}
\hbadness=99999

% Code Syntax Setup
\usepackage{listings}
\usepackage{xcolor} % untuk warna
\lstset{
basicstyle=\ttfamily\small,
frame=single,
breaklines=true,
backgroundcolor=\color{gray!10},
keywordstyle=\color{blue},
commentstyle=\color{gray},
stringstyle=\color{red},
tabsize=2,
captionpos=b
}

% Start of the Article
\begin{document}

% Cover Page
\pagenumbering{gobble}
\maketitle
\newpage

% Table of Content Page
\tableofcontents
\newpage
\pagenumbering{arabic}

% Main Content
\section{Business Models and Keeping Customers}

\subsection{KEY QUESTIONS}
\begin{itemize}
    \item What is a sustainable business model?
    \item How much capital do I need?
    \item How can I estimate when I'm going to run out of cash  or when I will reach positive cash flow?
    \item What are the available sources of capital?
    \item What financial analyses do I need for fund raising?
    \item What other information do investors typically require?
    \item When I succeed, how can I extract the financial value from my business?
\end{itemize}

\subsection{Business Models sebagai Narasi}
\begin{itemize}
    \item Business model berfungsi sebagai cerita yang menjelaskan bagaimana bisnis Anda dapat berhasil. Ini menggabungkan angka dan narasi untuk menunjukkan potensi kesuksesan.
    \item Contoh: WebVan memiliki narasi yang masuk akal untuk pengiriman bahan makanan, tetapi gagal karena biaya yang tinggi. Sebaliknya, Priceline memiliki narasi yang kurang masuk akal tetapi juga mengalami kerugian besar.
\end{itemize}

\subsection{Business Models sebagai Checklist}
\begin{itemize}
    \item Business models juga berfungsi sebagai checklist untuk membantu Anda memikirkan bagaimana bisnis Anda akan beroperasi dan menghasilkan uang.
    \item Pertanyaan penting yang harus dipertimbangkan:
          \begin{enumerate}
              \item What’s your value proposition? (Apa nilai yang Anda tawarkan kepada pelanggan?)
              \item How will you build your product or service? (Bagaimana Anda akan membangun produk atau layanan Anda?)
              \item How will you bring it to market? (Bagaimana Anda akan memasarkan produk atau layanan Anda?)
              \item How will you make money? (Bagaimana Anda akan menghasilkan uang?)
          \end{enumerate}
\end{itemize}

\subsection{Tipe-tipe Business Models}
\begin{itemize}
    \item Architecture Business Model: Model bisnis yang menghubungkan berbagai pihak, seperti eBay dan Uber, yang menghasilkan uang dari setiap transaksi tanpa memiliki produk.
    \item Disruptive Model: Model yang mencoba mengalahkan pesaing dengan menawarkan produk atau layanan yang lebih baik dan lebih murah, seperti Warby Parker yang menjual kacamata dengan harga lebih rendah.
    \item Value Chain Model: Model yang berfokus pada satu bagian dari rantai nilai dan melakukannya lebih baik daripada pesaing, seperti Peapod yang mengelola pengiriman bahan makanan.
\end{itemize}

\subsection{Pengertian Customer Lifetime Value (CLV)}
\begin{itemize}
    \item \textbf{Customer Lifetime Value (CLV)}: Nilai total yang diharapkan dari seorang pelanggan selama masa hubungan mereka dengan perusahaan.
    \item Definisi Istilah Teknis
          \begin{itemize}
              \item Annual Contribution: Jumlah uang yang dihasilkan dari pelanggan setiap tahun.
              \item Churn Rate: Persentase pelanggan yang berhenti menggunakan layanan dalam periode tertentu.
              \item Growth Rate: Persentase peningkatan kontribusi dari pelanggan setiap tahun.
              \item Discount Rate: Tingkat pengembalian yang diharapkan dari investasi, yang digunakan untuk menghitung nilai waktu dari uang.
          \end{itemize}
\end{itemize}

\subsection{Mengukur CLV}
\begin{itemize}
    \item CLV dihitung dengan mempertimbangkan beberapa faktor, yaitu kontribusi tahunan, churn rate, discount rate, dan growth rate. Rumus umumnya adalah:
          \begin{align*}
              CLV = \frac{\text{Annual Contribution}}{\text{Churn Rate} + \text{Discount Rate} - \text{Growth Rate}}
          \end{align*}
    \item Ada juga rumus cepatnya yang disebut Quick CLV atau QCLV, namun tidak seakurat CLV:
          \begin{align}
              QCLV = \frac{\text{Annual Contribution}}{\text{Churn Rate}}
          \end{align}
    \item Contoh Kasus: Sebuah perusahaan jasa keuangan yang kehilangan 20\% pelanggan setiap tahun, tetapi pelanggan yang bertahan meningkatkan kontribusi mereka sebesar 5\% setiap tahun.
    \item Langkah Pengukuran: Menggunakan data terbatas untuk memperkirakan nilai pelanggan dengan asumsi bahwa pelanggan memberikan kontribusi tahunan yang tetap.
\end{itemize}

\subsection{Faktor yang Mempengaruhi CLV}
\begin{itemize}
    \item \textbf{Annual Contribution}: Kontribusi tahunan dari pelanggan, misalnya \$250.
    \item \textbf{Churn Rate}: Tingkat kehilangan pelanggan, misalnya 20\%.
    \item \textbf{Growth Rate}: Tingkat pertumbuhan kontribusi tahunan, misalnya 5\%.
    \item \textbf{Discount Rate}: Tingkat diskonto atau nilai waktu dari uang, misalnya 10\%.
\end{itemize}

\subsection{Contoh Perhitungan CLV}
\begin{itemize}
    \item Kasus Pertama: Jika pelanggan memberikan \$250 setiap tahun tanpa kehilangan, CLV dihitung sebagai \$2,500.
    \item Kasus Kedua: Dengan 20\% kemungkinan kehilangan pelanggan, CLV menjadi sekitar \$833.
    \item Kasus Ketiga: Jika kontribusi meningkat 5\% setiap tahun, CLV menjadi sekitar \$1,000.
\end{itemize}

\subsection{Aplikasi CLV dalam Keputusan Bisnis}
\begin{itemize}
    \item Evaluasi Upaya Pemasaran: Perbandingan biaya akuisisi pelanggan dengan nilai pelanggan yang diharapkan.
    \item Strategi Retensi: Menghitung apakah mengurangi churn rate dari 20\% menjadi 18\% akan meningkatkan CLV secara signifikan.
    \item Perbandingan Pelanggan: Menggunakan CLV untuk membandingkan nilai pelanggan yang berbeda, seperti pelanggan prepaid dan postpaid dalam industri telekomunikasi.
\end{itemize}


\newpage

\section{Financing, Valuation, and Terms}
\newpage

\section{Private and Public Financing, and Calculating Breakeven}
\newpage

\section{Elements of the Pitch and Exit Strategies}
\newpage

\end{document}