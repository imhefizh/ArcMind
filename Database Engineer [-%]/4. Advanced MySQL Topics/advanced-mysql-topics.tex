% section | subsection | subsubsection | paragraph | subparagraph | verbatim | enumerate | item

% Just let it be an Article bro
\documentclass{article}

% Cover Metadata
\title{Advanced MySQL Topics}
\date{10 Oktober 2025}
\author{Maulana Hafidz Ismail}

% Image Setup
\usepackage{graphicx}
\usepackage{float}

% Link Setup
\usepackage{hyperref}

% Highlighting Setup
\usepackage{soul}
\usepackage{xcolor}
\sethlcolor{red!30}
\newcommand{\hlblue}[1]{\sethlcolor{cyan!30}\hl{#1}\sethlcolor{yellow}}

% Line Width
\sloppy
\setlength{\emergencystretch}{3em}
\hbadness=99999

% Code Syntax Setup
\usepackage{listings}
\usepackage{xcolor} % untuk warna
\lstset{
basicstyle=\ttfamily\small,
frame=single,
breaklines=true,
backgroundcolor=\color{gray!10},
keywordstyle=\color{blue},
commentstyle=\color{gray},
stringstyle=\color{red},
tabsize=2,
captionpos=b
}

% Start of the Article
\begin{document}

% Cover Page
\pagenumbering{gobble}
\maketitle
\newpage

% Table of Content Page
\tableofcontents
\newpage
\pagenumbering{arabic}

% Main Content
\section{Functions and Triggers}
\subsection{Pengantar Functions dan Stored Procedures}
\begin{itemize}
    \item Functions dan Stored Procedures memungkinkan Anda untuk menggunakan kembali kode dalam proyek database, sehingga tidak perlu mengetik kode yang sama berulang kali.
    \item Contoh penggunaan adalah pada perusahaan Lucky Shrub yang menggunakan metode ini untuk memeriksa data stok di tabel produk mereka.
    \item A function in MySQL is also called a stored function.
\end{itemize}

\subsection{Manfaat dan Organisasi Kode}
\begin{itemize}
    \item Tujuan utama dari functions dan stored procedures adalah untuk mengenkapsulasi kode, sehingga anda dapat memanggil blok kode untuk melakukan operasi tertenu dengan menggunakan nama pengidentifikasi.
    \item Keduanya membuat kode lebih konsisten, terorganisir, dan memperkenalkan reusability, yang memudahkan penggunaan dan pemeliharaan kode.
\end{itemize}

\subsection{Perbedaan Antara Functions dan Stored Procedures}
\begin{itemize}
    \item Functions selalu mengembalikan nilai, sedangkan stored procedures tidak selalu mengembalikan nilai.
    \item Function hanya dapat memiliki input parameters, sedangkan stored procedures dapat memiliki input dan output  parameters.
    \item Contoh sintaks untuk membuat stored procedure:
          \begin{lstlisting}[language=SQL, caption={}, captionpos=b]
    CREATE PROCEDURE procedure_name AS
        BEGIN
            -- logic here
        END;
    \end{lstlisting}
    \item Contoh sintaks untuk menggunakan function:
          \begin{lstlisting}[language=SQL, caption={}, captionpos=b]
    SELECT MOD(x, y)
    \end{lstlisting}
\end{itemize}

\subsection{Variabel dalam MySQL}
\begin{itemize}
    \item Variabel adalah placeholder yang menyimpan nilai yang dapat berubah sesuai kebutuhan query. Variabel digunakan untuk mengoper nilai antara pernyataan SQL atau antara prosedur dan pernyataan SQL.
    \item Variabel dapat dibuat di dalam atau di luar stored procedure dan di dalam atau di luar pernyataan SELECT.
    \item Contoh sintaks untuk membuat variabel:
          \begin{lstlisting}[language=SQL, caption={}, captionpos=b]
    SET @variabel_name = value;
    \end{lstlisting}
\end{itemize}

\subsection{Deklarasi Variabel}
\begin{itemize}
    \item DECLARE digunakan untuk mendeklarasikan variabel lokal di dalam stored procedure. Variabel ini hanya dapat digunakan dalam konteks prosedur tersebut. Contoh:
          \begin{lstlisting}[language=SQL, caption={}, captionpos=b]
    DECLARE variable_name datatype;
    \end{lstlisting}
    \item SET @ digunakan untuk mendapatkan variabel sesi (session variable) yang dapat diakses di seluruh sesi MySQL Variabel ini diawali dengan simbol @. Contoh:
          \begin{lstlisting}[language=SQL, caption={}, captionpos=b]
    SET @variable_name = value;
    \end{lstlisting}
\end{itemize}

\subsection{Mengatur Nilai Variabel}
\begin{enumerate}
    \item SET digunakan untuk memberikann nilai kepada variabel yang telah dideklarasikan. Ini adalah cara yang umum untuk menginisialisasi variabel. Contoh:
          \begin{lstlisting}[language=SQL, caption={}, captionpos=b]
    SET variable_name = value;
    \end{lstlisting}
    \item Operator (:=) digunakan untuk menetapkan nilai ke variabel, baik untuk variabel lokal yang dideklarasikan dengan DECLARE maupun untuk variabel sesi yang diawali dengan @. Contoh:
          \begin{lstlisting}[language=SQL, caption={}, captionpos=b]
    variable_name := value;
    \end{lstlisting}
\end{enumerate}

\subsection{Contoh Penggunaan}
\begin{lstlisting}[language=SQL, caption={}, captionpos=b]
CREATE PROCEDURE example_procedure()
BEGIN
    DECLARE total_cost DECIMAL(10,2);
    SET total_cost = 100.00;
    SET @discount = 10;
    total_cost := total_cost - @discount;
END;
\end{lstlisting}

% BELUM SELESAI MIGRASI
\section{Database Optimization}
% BELUM MIGRASI




\end{document}